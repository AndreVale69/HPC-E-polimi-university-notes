\subsection{Metodi iterativi per sistemi lineari}

Un \definition{metodo iterativo} per la risoluzione del sistema lineare $Ax = b$ con:
\begin{itemize}
    \item $A \in \mathbb{R}^{n \times n}$
    \item $b \in \mathbb{R}^{n}$
    \item $x \in \mathbb{R}^{n}$
    \item $\det\left(A\right) \ne 0$
\end{itemize}
Consiste nel costruire una successione di vettori del tipo:
\begin{equation*}
    \left\{\mathbf{x}^{\left(k\right)}, \: k \ge 0\right\}
\end{equation*}
Di $\mathbb{R}^{n}$ che \textbf{converge} alla soluzione esatta $\mathbf{x}$, ossia tale che:
\begin{equation}
    \displaystyle\lim_{k \rightarrow \infty} \mathbf{x}^{\left(k\right)} = \mathbf{x}
\end{equation}
Per un qualunque vettore iniziale $\mathbf{x}^{\left(0\right)} \in \mathbb{R}$, ossia la \textbf{convergenza non deve dipendere dalla scelta di} $\mathbf{x}^{\left(0\right)}$.

\highspace
L'esigenza di introdurre i metodi iterativi sorge nel momento in cui si ragiona sulla quantità di tempo spesa da un calcolatore per eseguire la fattorizzazione LU su matrici di grandi dimensioni. Difatti, con matrici con ordini di $10^{7}$, sono necessari circa 11 giorni.