\subsection{Sistema di equazioni non lineari}

\subsubsection{Metodo di Newton}

Si consideri il problema della ricerca degli zeri del sistema non lineare $\mathbf{f}\left(\mathbf{x}\right) = \mathbf{0}$, dove $\mathbf{f}: \mathbb{R}^{n} \rightarrow \mathbb{R}^{n}$. Definendo: 
\begin{itemize}
	\item $\mathbf{x} = \left(x_{1}, \dots, x_{n}\right)^{T}$
	\item $\mathbf{f} = \left(f_{1}, \dots, f_{n}\right)^{T}$ con $f_{1}, \dots, f_{n}$ funzioni $\mathbb{R}^{n} \rightarrow \mathbb{R}$
\end{itemize}
Il problema si può riscrivere nel seguente modo:
\begin{equation*}
	\begin{cases}
		f_{1}\left(x_{1}, \dots, x_{n}\right) = 0 \\
		f_{2}\left(x_{1}, \dots, x_{n}\right) = 0 \\
		\vdots \\
		f_{n}\left(x_{1}, \dots, x_{n}\right) = 0
	\end{cases}
\end{equation*}
Il metodo di Newton per sistemi non lineari è il seguente. Dato $\mathbf{x}^{\left(0\right)} \in \mathbb{R}^{n}$, per $k \ge 0$:
\begin{enumerate}
	\item Risolvere il sistema lineare:
	\begin{equation*}
		J\left(\mathbf{x}^{k}\right)\delta\mathbf{x}^{\left(k\right)} = -\mathbf{f}\left(\mathbf{x}^{\left(k\right)}\right)
	\end{equation*}
	
	\item Ponendo:
	\begin{equation*}
		\mathbf{x}^{\left(k+1\right)} = \mathbf{x}^{\left(k\right)} + \delta\mathbf{x}^{\left(k\right)}
	\end{equation*}
\end{enumerate}
Dove, per un generico punto $\mathbf{y}$, $J\left(\mathbf{y}\right)$ è la matrice Jacobiana della funzione $\mathbf{f}$ e consiste in una matrice $\mathbb{R}^{n} \times \mathbb{R}^{n}$ le cui componenti sono:
\begin{equation*}
	J_{il}\left(\mathbf{y}\right) = \dfrac{\partial f_{i}\left(\mathbf{y}\right)}{\partial y_{l}} \hspace{2em} i,l = 1, \dots, n
\end{equation*}
Si osservi che:
\begin{enumerate}
	\item L'applicazione del metodo di Newton richiede ad ogni iterazione la soluzione di un sistema lineare con matrice $A^{\left(k\right)} = J\left(\mathbf{x}^{\left(k\right)}\right)$.
	
	\item È possibile dimostrare che se:
	\begin{enumerate}
		\item $\mathbf{x}^{\left(0\right)}$ è \dquotes{sufficientemente} vicino alla soluzione $\mathbf{\alpha}$
		
		\item $J\left(\mathbf{x}^{\left(k\right)}\right)$ è una matrice non singolare con $k = 0, 1, \dots$
	\end{enumerate}
	 Allora il metodo di Newton converge con ordine 2.
\end{enumerate}

\newpage

Si consideri il sistema non lineare seguente la cui incognite è il vettore $\mathbf{x} = \left[x_{1}, x_{2}\right]$:
\begin{equation*}
	\begin{cases}
		-x_{1} + e^{3x_{2}} = 1 \\
		-x_{1} + x_{1}x_{2}^{2} = -2
	\end{cases}
\end{equation*}
\begin{enumerate}
	\item Scrivere Il sistema non lineare sotto la forma:
	\begin{equation*}
		\mathbf{f}\left(\mathbf{x}\right) = \left[f_{1}\left(\mathbf{x}\right), f_{2}\left(\mathbf{x}\right)\right]^{T} = \mathbf{0}
	\end{equation*}
	Rappresentare $\mathbf{f}$ mediante una funzione Matlab di tipo \emph{anonymous function} che, ricevuto in input un vettore colonna di 2 elementi, restituisce un vettore colonna di 2 elementi contenente la valutazione di $\mathbf{f}\left(\mathbf{x}\right)$. Analogamente, calcolare la matrice Jacobiana $J\left(\mathbf{x}\right)$ di $\mathbf{f}\left(\mathbf{x}\right)$ e scrivere la \emph{anonymous function} che restituisca tale matrice.
	
	Dunque, il sistema da risolvere è:
	\begin{equation*}
		\mathbf{f}\left(\mathbf{x}\right) = \begin{bmatrix}
			-x_{1} + e^{3x_{2}}-1 \\
			-x_{1} + x_{1}x_{2}^{2} + 2
		\end{bmatrix} = \mathbf{0}
	\end{equation*}
	Dove $\mathbf{x} = \left[x_{1}, x_{2}\right]^{T}$. La matrice Jacobiana è la seguente:
	\begin{equation*}
		J\left(\mathbf{x}\right) = \dfrac{\partial \mathbf{f}}{\partial \mathbf{x}} = \begin{bmatrix}
			-1 & 3e^{3x_{2}} \\
			-1+x_{2}^{2} & 2x_{1}x_{2}
		\end{bmatrix}
	\end{equation*}
	Si introduce il vettore $\mathbf{f}$ e la matrice $J$ utilizzando i seguenti comandi:
	\lstinputlisting[language=MATLAB]{code/risoluzione-di-sistemi-di-equazioni-non-lineari/newton_1.m}
	
	
	\item Implementare il metodo di Newton per sistemi. L'intestazione della funzione sarà ad esempio:
	\begin{center}
		\texttt{[xvect, it] = newtonsys(fun, Jf, x0, toll, nmax)}
	\end{center}
	Con ovvio significato dei parametri di ingresso e di uscita. Si utilizzi un criterio d'arresto basato sulla norma della differenza tra due iterate successive.
	
	\emph{Suggerimento}: si utilizzi il comando $\backslash$ per la risoluzione dei sistemi lineari.
	
	\lstinputlisting[language=MATLAB]{code/risoluzione-di-sistemi-di-equazioni-non-lineari/newtonsys.m}
	
	
	\item Utilizzare la funzione scritta al punto precedente per calcolare la soluzione del sitema non lineare proposto. Utilizzare una tolleranza di \texttt{toll = 1e-6}, un vettore iniziale \texttt{x0 = [1;0]}. Fornire il risultato e il numero di iterazioni effettuate.
	
	Si utilizza la funzione come richiesto.
	
	\lstinputlisting[language=MATLAB]{code/risoluzione-di-sistemi-di-equazioni-non-lineari/newton_2.m}
\end{enumerate}