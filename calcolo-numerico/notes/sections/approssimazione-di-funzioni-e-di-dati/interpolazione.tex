\section{Approssimazione di funzioni e di dati}

\subsection{Interpolazione}

In molte applicazioni concrete si conosce una funzione solo attraverso i suoi valori in determinati punti. Si supponga di conoscere $n+1$ coppie di valori $\left\{x_{i}, y_{i}\right\}$ con $i = 0, \dots, n$, dove i punti $x_{i}$, tutti distinti, vengono chiamati \definition{nodi}.

\highspace
In tal caso, può apparire naturale richiedere che la funzione approssimante $\tilde{f}$ soddisfi le seguenti uguaglianze:
\begin{equation}\label{eq: interpolatore}
	\tilde{f}\left(x_{i}\right) = y_{i} \hspace{2em} i = 0, 1, \dots, n
\end{equation}
Una tale funzione $\tilde{f}$ viene chiamata \definition{interpolatore} dell'insieme di dati $\left\{y_{i}\right\}$ e le equazioni del tipo \ref{eq: interpolatore} sono le \textbf{condizioni di interpolazione}.

\highspace
Esistono vari tipi di interpolatori:
\begin{itemize}
	\item L'\definition{interpolatore polinomiale}:
	\begin{equation*}
		\tilde{f}\left(x\right) = a_{0} + a_{1}x + a_{2}x^{2} + \cdots + a_{n}x^{n}
	\end{equation*}
	
	\item L'\definition{interpolatore trigonometrico}:
	\begin{equation*}
		\tilde{f}\left(x\right) = a_{-M} e^{-i M x} + \cdots + a_{0} + \cdots + a_{M} e^{i M x}
	\end{equation*}
	Dove $M$ è un intero pari a $\frac{n}{2}$ se $n$ è pari, $\frac{\left(n+1\right)}{2}$ se $n$ è dispari, e $i$ è l'unità immaginaria.
	
	\item L'\definition{interpolatore razionale}:
	\begin{equation*}
		\tilde{f}\left(x\right) = \dfrac{
			a_{0} + a_{1}x + \cdots + a_{k}x^{k}
		}{
			a_{k+1} + a_{k+2}x + \cdots + a_{k+n+1} x^{n}
		}
	\end{equation*}
\end{itemize}