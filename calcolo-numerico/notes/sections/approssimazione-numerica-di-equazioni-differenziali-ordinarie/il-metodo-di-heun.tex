\subsection{Il metodo di Heun}

Partendo dal metodo di Crank-Nicolson introdotto nel paragrafo \ref{subsection: il metodo di Crank-Nicolson} a pagina \pageref{subsection: il metodo di Crank-Nicolson}:
\begin{equation*}
	u_{n+1} = u_{n} + \dfrac{h}{2}\left(f\left(t_{n}, u_{n}\right) + f\left(t_{n+1}, u_{n+1}\right)\right)
\end{equation*}
E al posto della variabile $u_{n+1}$ a destra si introduce la sua approssimazione ottenuta con il metodo di Eulero in avanti, si ottiene il \definition{metodo di Heun}:
\begin{equation}
	u_{n+1} = u_{n} + \dfrac{h}{2}\left(f\left(t_{n}, u_{n}\right) + f\left(t_{n+1}, u_{n} + hf\left(t_{n}, u_{n}\right)\right)\right)
\end{equation}
Esso è un \textbf{metodo esplicito} con \textbf{condizione di assoluta stabilità uguale a quella del metodo di Eulero in avanti} (si veda a pagina \pageref{Assoluta stabilità: metodo di Eulero in avanti} l'assoluta stabilità di tale metodo).