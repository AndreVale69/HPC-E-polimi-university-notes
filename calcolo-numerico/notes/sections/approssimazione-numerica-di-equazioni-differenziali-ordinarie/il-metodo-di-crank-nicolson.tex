\subsection{Il metodo di Crank-Nicolson}

Dato il problema di Cauchy:
\begin{equation*}
	\begin{cases}
		y'\left(t\right) = f\left(t, y\left(t\right)\right) \hspace{2em} \forall t \in I
		y\left(t_{0}\right) = y_{0}
	\end{cases}
\end{equation*}
Dal teorema fondamentale del calcolo integrale limitato all'intervallo $\left[t_{n}, t_{n+1}\right]$ si ottiene:
\begin{equation*}
	y\left(t_{n+1}\right) = y\left(t_{n}\right) + \displaystyle\int_{t_{n}}^{t_{n+1}} f\left(t, y\left(t\right)\right) \: \mathrm{d}t
\end{equation*}
Adesso si applica una formula di quadratura a disposizione (punto medio composita, trapezi composita, Simpson composita) per approssimare l'integrale a destra dell'uguale e ottenere così un metodo di approssimazione per il problema di Cauchy.

\highspace
Ed ecco che nasce il \definition{metodo di Crank-Nicolson (CN)} applicando la formula dei trapezi composita (par. \ref{subsection: formula dei trapezi composita}, pag. \pageref{subsection: formula dei trapezi composita}) considerando solo un intervallo $\left[t_{n}, t_{n+1}\right]$:
\begin{equation}
	u_{n+1} = u_{n} + \dfrac{h}{2}\left(f\left(t_{n}, u_{n}\right) + f\left(t_{n+1}, u_{n+1}\right)\right)
\end{equation}
Esso è un \textbf{metodo implicito} e richiede ad ogni passo $n$ di determinare la radice della funzione:
\begin{equation*}
	g\left(x\right) = x - \dfrac{h}{2} f\left(t_{n+1}, x\right) - u_{n} - \dfrac{h}{2} f\left(t_{n}, u_{n}\right)
\end{equation*}

\highspace
Riguardo l'\textbf{assoluta stabilità}, si ha:
\begin{equation*}
	u_{n+1} = u_{n} + \dfrac{h \lambda}{2}\left(u_{n} + u_{n+1}\right) \: \rightarrow \: u_{n+1} = \dfrac{2+h\lambda}{2-h\lambda} u_{n}
\end{equation*}
Di conseguenza si ottiene:
\begin{equation*}
	C_{AS} = \left|\dfrac{2+h\lambda}{2-h\lambda}\right|
\end{equation*}
Che è sempre minore di 1. Per cui il metodo di Crank-Nicolson è \textbf{incondizionatamente assolutamente stabile}.