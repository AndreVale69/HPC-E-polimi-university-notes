\section{Introduction}

\begin{definitionbox}
    \definition{Operations Research (OR)}, often shortened to the initialism \texttt{OR}, is the branch of mathematics in which \textbf{mathematical models} and \textbf{quantitative methods} (e.g. optimization, game theory, simulation) are \textbf{used to analyze complex decision-making problems} and \textbf{find (near-)optimal solutions}.
\end{definitionbox}

\highspace
The overall and primary \emph{goal} is to \emph{help make better decisions}.

\highspace
OR can be seen as an interdisciplinary field at the intersection of applied mathematics, computer science, economics, and industrial engineering.

\highspace
Operations research is often concerned with \textbf{determining the extreme values of some real-world objective}: the \emph{maximum} (of profit, performance, or yield) or \emph{minimum} (of loss, risk, or cost). Originating in military efforts before World War II, its techniques have grown to concern problems in a variety of industries.\cite{wikipediaOperationsResearch}

\longline

\subsection{Decision-making problems}

Decision-making problems are analyzed using mathematical models and quantitative methods.

\begin{definitionbox}
    \definition{Decision-making problems} are problems in which we must \textbf{choose} a (feasible) \textbf{solution among a large number of alternatives based on one or several criteria}.
\end{definitionbox}

\highspace
Some practical \example{examples} include network design, shortest paths, staff scheduling, and service management.

\highspace
In other words, they are complex decision-making problems that are \textbf{addressed through a mathematical modeling approach} (mathematical models, algorithms, and computer implementations).