\subsection{Mathematical programming/optimization}\label{subsection: Mathematical programming/optimization}

\definition{Mathematical Optimization} or \definition{Mathematical Programming} is the \textbf{selection of a best element}, with regard to some criteria, \textbf{from some set of available alternatives}.

In the more general approach, an optimization problem consists of \textbf{maximizing} or \textbf{minimizing a real function} by systematically choosing input values from within an allowed set and computing the value of the function. The generalization of optimization theory and techniques to other formulations constitutes a large area of applied mathematics.

\highspace
\begin{flushleft}
    \textcolor{Green3}{\faIcon{question-circle} \textbf{Okay, how is it defined mathematically?}}
\end{flushleft}
Mathematical Operation problems belong to the category of decision-making problems. They are characterized by a \textbf{single decision maker}, a \textbf{single objective}, and \textbf{reliable parameter estimates}. In mathematical language, we can say:
\begin{equation*}
    \opt f\left(\mathbf{x}\right) \hspace{1em} \text{with} \hspace{1em} \mathbf{x} \in X \hspace{1em} \text{and} \hspace{1em} \opt = \begin{Bmatrix}
        \min \\ \max
    \end{Bmatrix}
\end{equation*}
Where:
\begin{itemize}
    \item $\mathbf{x} \in \mathbb{R}^{n}$ \textbf{decision variables}. They are numerical variables whose values identify a solution of the problem.

    \item $X \subseteq \mathbb{R}^{n}$ \textbf{feasible region}. Distinguishes between feasible and infeasible solutions (via constraints):
    \begin{equation*}
        X = \left\{\mathbf{x} \in \mathbb{R}^{n} \: : \: g_{i}\left(\mathbf{x}\right) \begin{Bmatrix}
        = \\ \le \\ \ge
        \end{Bmatrix} 0, i = 1, \dots, m\right\}
    \end{equation*}

    \item $f: X \rightarrow \mathbb{R}$ \textbf{objective function}.Expresses in quantitative terms the value or cost of each feasible solution.
\end{itemize}
Note an interesting observation:
\begin{equation*}
    \max\left\{f\left(\mathbf{x}\right): \: \mathbf{x} \in X\right\} = -\min\left\{-f\left(\mathbf{x}\right): \: \mathbf{x} \in X\right\}
\end{equation*}

\highspace
\begin{flushleft}
    \textcolor{Green3}{\faIcon{question-circle} \textbf{More specifically, how can we solve these problems?}}
\end{flushleft}
It depends on how hard the given problem is to solve.
\begin{itemize}
    \item The problem has an \textbf{easy}/\textbf{medium level} of complexity. It makes sense to use the \definition{Global Optima} technique. It consists in finding a feasible solution that is \textbf{globally optimum}, then a vector $\mathbf{x}^{*} \in X$ such that:
    \begin{equation*}
        \begin{array}{lll}
            f\left(\mathbf{x}^{*}\right) \le f\left(\mathbf{x}\right) & \forall \mathbf{x} \in X & \text{if } \opt = \min \\ [.5em]
            f\left(\mathbf{x}^{*}\right) \ge f\left(\mathbf{x}\right) & \forall \mathbf{x} \in X & \text{if } \opt = \max
        \end{array}
    \end{equation*}
    Unfortunately, this method is not perfect and it may happen that the given problem occurs:
    \begin{itemize}
        \item Is \textbf{infeasible}, so the feasible region is empty: $X = \emptyset$.
        \item Is \textbf{unbounded}: $\forall c \in \mathbb{R}$, $\exists\mathbf{x}_{c} \in X$ such that $f\left(\mathbf{x}_{c}\right) \le c$ or $f\left(\mathbf{x}_{c}\right) \ge c$.
        \item Has a \textbf{single optimal solution}.
        \item Has a \textbf{large number} (even an infinite number) \textbf{of optimal solutions} (with the same optimal value!).
    \end{itemize}

    \item The problem has a \textbf{difficult}/\textbf{hard level} of complexity. Then the \definition{Local Optima} is the best choice. It consists in finding a feasible solution that is \textbf{local optimum} (main different against global optima technique), then a vector $\widehat{\mathbf{x}} \in X$ such that:
    \begin{equation*}
        \begin{array}{lll}
            f\left(\widehat{\mathbf{x}}\right) \le f\left(\mathbf{x}\right) & \forall \mathbf{x} \text{ with } \mathbf{x} \in X \text{ and } \left|\left| \mathbf{x} - \widehat{\mathbf{x}} \right|\right| \le \varepsilon & \text{if } \opt = \min \\ [.5em]
            f\left(\widehat{\mathbf{x}}\right) \ge f\left(\mathbf{x}\right) & \forall \mathbf{x} \text{ with } \mathbf{x} \in X \text{ and } \left|\left| \mathbf{x} - \widehat{\mathbf{x}} \right|\right| \le \varepsilon & \text{if } \opt = \max
        \end{array}
    \end{equation*}
    For an appropriate value $\varepsilon > 0$.

    In this case, it may happen that the given \textbf{problem has many local optima}.
\end{itemize}

\highspace
\begin{flushleft}
    \textcolor{Red2}{\faIcon{bookmark} \textbf{Categories}}
\end{flushleft}
A Mathematical Programming can be \textbf{categorized} depending on the feasible region:
\begin{itemize}
    \item \definition{Linear Programming (LP)}. The function $f$ is linear:
    \begin{equation*}
        X = \left\{\mathbf{x} \in \mathbb{R}^{n} \: : \: g_{i}\left(\mathbf{x}\right) \begin{Bmatrix}
        = \\ \le \\ \ge
        \end{Bmatrix} 0, \: i = 1, \dots, m\right\} \text{ with } g_{i} \text{ linear } \forall i
    \end{equation*}
    An \example{example} is the \emph{production planning}.
    
    \item \definition{Integer Linear Programming (ILP)}. The function $f$ is linear:
    \begin{equation*}
        X = \left\{\mathbf{x} \in \mathbb{R}^{n} \: : \: g_{i}\left(\mathbf{x}\right) \begin{Bmatrix}
        = \\ \le \\ \ge
        \end{Bmatrix} 0, \: i = 1, \dots, m\right\} \cap \mathbb{Z}^{n} \text{ with } g_{i} \text{ linear } \forall i
    \end{equation*}
    An \example{example} is the \emph{portfolio selection} (finance). As we can see, the ILP technique is identical to LP with additional integrality constraints on the variables.
    
    \item \definition{Nonlinear Programming (NLP)}. The function $f$ is convex/regular or non convex/regular:
    \begin{equation*}
        X = \left\{\mathbf{x} \in \mathbb{R}^{n} \: : \: g_{i}\left(\mathbf{x}\right) \begin{Bmatrix}
        = \\ \le \\ \ge
        \end{Bmatrix} 0, \: i = 1, \dots, m\right\}
    \end{equation*}
    With $g_{i}$ convex/regular or not convex/regular $\forall i$.
    
    An \example{example} is the \emph{facility location} (with $g_{i}$ convex).
\end{itemize}

\newpage

\begin{flushleft}
    \textcolor{Red2}{\faIcon{bookmark} \textbf{History of Mathematical Programming}}
\end{flushleft}
It is correct to report the history of mathematical programming:
\begin{itemize}
    \item[1826/27] Joseph Fourier presents a method to solve systems of linear inequalities (Fourier-Motzkin) and discusses some LPs with 2-3 variables.
    
    \item[1939] Leonid Kantorovitch lays the bases of LP (Nobel prize 1975).
    
    \item[1947] George Dantzig proposes independently LP and invents the simplex algorithm.

    \item[1958] Ralph Gomory proposes a cutting plane method for ILP problems.
\end{itemize}