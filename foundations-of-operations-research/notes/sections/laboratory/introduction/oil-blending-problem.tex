\subsubsection{Oil blending problem}

A refinery has to blend 4 types of oil to obtain 3 types of gasoline. The following table reports the available quantity of oil of each type (in barrels) and the respective unit cost (Euro per barrel):

\begin{table}[!htp]
    \centering
    \begin{tabular}{@{} c c c @{}}
        \toprule
        \textbf{Oil type} & \textbf{Cost} & \textbf{Availability} \\
        \midrule
        1 & 9   & 5000 \\
        2 & 7   & 2400 \\
        3 & 12  & 4000 \\
        4 & 6   & 1500 \\
        \bottomrule
    \end{tabular}
\end{table}

\noindent
Blending constraints are to be taken into account, since each type of gasoline must contain at least a specific, predefined, quantity of each type of oil, as indicated in the next table. The unit revenue of each type of gasoline (Euro per barrel) is also indicated:

\begin{table}[!htp]
    \centering
    \begin{tabular}{@{} c c c @{}}
        \toprule
        \textbf{Gasoline type} & \textbf{Requirements} & \textbf{Revenue} \\
        \midrule
        A & $\geq$ 20\% of type 2 & 12 \\
        A & $\leq$ 30\% of type 3 & 12 \\
        B & $\geq$ 40\% of type 3 & 18 \\
        C & $\leq$ 50\% of type 2 & 10 \\
        \bottomrule
    \end{tabular}
\end{table}

\highspace
\begin{flushleft}
    \textcolor{Green3}{\faIcon{square-root-alt} \textbf{Oil blending problem formulation}}
\end{flushleft}
\begin{itemize}
    \item Sets
    \begin{itemize}
        \item $I$: set of oil types
        \item $J$: set of gasoline types
    \end{itemize}

    \item Parameters
    \begin{itemize}
        \item $c_i$: unit cost for oil of type $i \in I$ 
        \item $b_i$: availability of oil of type $i \in I$
        \item $r_j$: price of gasoline of type $i \in I$
        \item $b_i$: minimum quantity of nutrient $i \in I $ required
        \item $q_{ij}^{\max}$: maximum quantity (percentage) of oil of type $i \in I$ for gasoline of type $j \in J$
        \item $q_{ij}^{\min}$: minimum quantity (percentage) of oil of type $i \in I$ for gasoline of type $j \in J$
    \end{itemize}

    \newpage

    \item Variables
    \begin{itemize}
        \item $x_{ij}$: units of oil of type $i \in I$ used for gasoline of type $j \in J$
        \item $y_j$: units of gasoline of type $j \in J$ that are produced
    \end{itemize}

    \item Model
    \begin{equation*}
        \begin{array}{llll}
            \max & \displaystyle\sum_{j \in J}r_j y_j-\displaystyle\sum_{i \in I,j\in J} c_{j} x_{ij} \qquad & & \text{(revenue)} \\ [1.5em]
            \textrm{s. t.} & \displaystyle\sum_{j \in J}x_{ij} \leq b_i & i \in I & \text{(availability)} \\ [1.5em]
            & \displaystyle\sum_{i \in I}x_{ij}  = y_j & j \in J & \text{(conservation)} \\ [1.4em]
            & x_{ij} \leq q_{ij}^{\max}y_j & i \in I, j \in J & \text{(maximum qty)} \\ [.8em]
            & x_{ij} \geq q_{ij}^{\min}y_j & i \in I, j \in J & \text{(minimum qty)} \\ [.8em]
            & x_{ij},y_j \geq 0 &  i \in I,j \in J & \text{(nonnegativity)}
          \end{array}
    \end{equation*}
\end{itemize}

\highspace
\begin{flushleft}
    \textcolor{Green3}{\faIcon{laptop-code} \textbf{Oil blending problem implementation}}
\end{flushleft}
\begin{enumerate}
    \item Write the dataset specified by input:
    \begin{lstlisting}[language=Python]
# Set of oil types
I = range(4)

# Set of gasoline types
J = {'A', 'B', 'C'}

# Unit cost for oil of type i 
c = {0:9, 1:7, 2:12, 3:6}

# Availability of oil type i
b = {0:5000, 1:2400, 2:4000, 3:1500}

# Price of gasoline of type j
r = {'A':12, 'B':18, 'C':10}

# Maximum quantity (percentage) of oil
q_max = {}
for i in I:
    for j in J:
        q_max[(str(i), j)] = 1
q_max[('2', 'A')] = 0.3
q_max[('1', 'C')] = 0.5

# Minimum quantity (percentage) of oil
q_min = {}
for i in I:
    for j in J:
        q_min[(str(i), j)] = 0
q_min[('1', 'A')] = 0.2
q_min[('2', 'B')] = 0.4\end{lstlisting}

    \item Now we create an empty model and add the variables:
    \begin{lstlisting}[language=Python]
# Define a model
model2 = mip.Model()

# Define variables
x = {
    (str(i), j): model2.add_var(name=str(i)+','+j, lb=0)
    for i in I for j in J
}
y = {
    j: model2.add_var(name=j, lb=0) for j in J
}\end{lstlisting}

    \item Let us write the objective function: in the general case, it can be written as a sum over the set of models.
    \begin{lstlisting}[language=Python]
# Define the objective function
model2.objective = mip.maximize(
    mip.xsum(y[j]*r[j] for j in J) - 
    mip.xsum(c[i]*x[str(i), j] for i in I for j in J)
)\end{lstlisting}

    \item The constraints can be generated in loops:
    \begin{lstlisting}[language=Python]
# CONSTRAINTS
# Availability constraint
for i in I:
    model2.add_constr(mip.xsum(x[str(i), j] for j in J) <= b[i])

# Conservation constraint
for j in J:
    model2.add_constr(mip.xsum(x[str(i), j] for i in I) == y[j])
# Maximum quantity
for i in I:
    for j in J:
        model2.add_constr(x[str(i), j] <= q_max[str(i),j]*y[j])
# Minimum quantity
for i in I:
    for j in J:
        model2.add_constr(x[str(i), j] >= q_min[str(i),j]*y[j])\end{lstlisting}

\newpage
    \item The model is complete. Let's solve it and print the optimal solution.
    \begin{lstlisting}[language=Python]
# Optimizing command
print(model2.optimize())
# OptimizationStatus.OPTIMAL

# Optimal objective function value
print(model2.objective.x)
# 96000.0

# Printing the variables values
for i in model2.vars:
    print(i.name, i.x)
# 0,C   0.0
# 0,A   500.0
# 0,B   4500.0
# 1,C   0.0
# 1,A   2400.0
# 1,B   0.0
# 2,C   0.0
# 2,A   0.0
# 2,B   4000.0
# 3,C   0.0
# 3,A   0.0
# 3,B   1500.0
# C     0.0
# A     2900.0
# B     10000.0\end{lstlisting}
\end{enumerate}