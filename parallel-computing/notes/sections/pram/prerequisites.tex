\section{PRAM}

\subsection{Prerequisites}

Before we introduce the PRAM model, we need to cover some useful topics.
\begin{itemize}
    \item A \definition{Machine Model} describes a \dquotes{machine}. It gives a value to the operations on the machine. It is necessary because: it makes it easy to deal with algorithms; it achieves complexity bounds; it analyses maximum parallelism.

    \item A \definition{Random Access Machine (RAM)} is a model of computation that describes an abstract machine in the general class of register machines. Some features are:
    \begin{itemize}
        \item \textbf{Unbounded} number of local memory cells;
        \item Each memory cell can hold an integer of \textbf{unbounded} size;
        \item Instruction set includes simple operations, data operations, comparator, branches;
        \item All operations take \textbf{unit time};
        \item The definition of \textbf{time complexity} is the number of instructions executed;
        \item The definition of \textbf{space complexity} is the number of memory cells used.
    \end{itemize}
\end{itemize}