\section{PRAM}

\subsection{Prerequisites}

Before we introduce the PRAM model, we need to cover some useful topics.
\begin{itemize}
    \item A \definition{Machine Model} describes a \dquotes{machine}. It gives a value to the operations on the machine. It is necessary because: it makes it easy to deal with algorithms; it achieves complexity bounds; it analyses maximum parallelism.

    \item A \definition{Random Access Machine (RAM)} is a model of computation that describes an abstract machine in the general class of register machines. Some features are:
    \begin{itemize}
        \item \textbf{Unbounded} number of local memory cells;
        \item Each memory cell can hold an integer of \textbf{unbounded} size;
        \item Instruction set includes simple operations, data operations, comparator, branches;
        \item All operations take \textbf{unit time};
        \item The definition of \textbf{time complexity} is the number of instructions executed;
        \item The definition of \textbf{space complexity} is the number of memory cells used.
    \end{itemize}
\end{itemize}

\longline

\subsection{Definition}

\begin{definitionbox}[: PRAM]
    A \definitionWithSpecificIndex{parallel random-access machine (parallel RAM or PRAM)}{Parallel Random-Access Machine (parallel RAM or PRAM)}{} is a \textbf{shared-memory abstract machine}. As its name indicates, the PRAM is intended as the parallel-computing analogy to the random-access machine (RAM) (\underline{not} to be confused with random-access memory). In the same way that the RAM is used by sequential-algorithm designers to model algorithmic performance (such as time complexity), the \textbf{PRAM is used by parallel-algorithm designers to model parallel algorithmic performance} (such as time complexity, where the number of processors assumed is typically also stated).
\end{definitionbox}

\noindent
The PRAM model has many interesting features:
\begin{itemize}
    \item \textbf{Unbounded collection of RAM processors} ($P_{0}$, $P_{1}$, and so on);
    \item Processors don't have tape;
    \item Each processor has \textbf{unbounded registers};
    \item \textbf{Unbounded} collection of \textbf{share memory cells};
    \item All \textbf{processors} can \textbf{access all memory cells in unit time};
    \item All \textbf{communication via shared memory}.
\end{itemize}