\subsection{Shared Address Space Model}

We give a general introduction to memory in the chapter \ref{subsection: Accessing Memory}. In parallel computing theory, \textbf{each thread communicates with other threads using read/write operations}. These instructions operate on a special area called the \definition{Shared Address Space} (also called \definition{Shared Variables}).

\highspace
\begin{definitionbox}[: Shared Address Space]
    The \definition{Shared Address Space} view of a parallel platform supports a common data space that is accessible to all processors. Processors interact by modifying data objects stored in this shared-address-space.\cite{kumar1994introduction}
\end{definitionbox}

\highspace
Now the first and trivial question should be: a powerful tool is the possibility to allow communication between threads, but \emph{how can we guarantee that two or more threads accessing the same resource do not create well known problems, such as} \href{https://en.wikipedia.org/wiki/Race_condition#Data_race}{race condition}\emph{?}

This property, commonly called \textbf{mutual exclusion} or \textbf{atomic operation}, can be guaranteed with some techniques:
\begin{itemize}
    \item \textbf{Lock/Unlock mutex around a critical section}:
    \begin{lstlisting}[language=c++]
Lock lock_variable;

// some operations, such as spawn of threads

lock_variable.lock();
// critical section
lock_variable.unlock();\end{lstlisting}
    
    \item Some languages have first-class \textbf{support for atomicity of code blocks}:
    \begin{lstlisting}[language=c++]
atomic {
    // critical section
}\end{lstlisting}

    \item Intrinsics for \textbf{hardware-supported atomic read-modify-write operations}:
    \begin{lstlisting}[language=c++]
atomicAdd(x, 10);\end{lstlisting}
\end{itemize}
The shared address space \textbf{requires hardware support to be efficiently implemented}. The main idea is that \textbf{each processor can directly reference the contents of any memory location}. Some interesting examples that can be explored in depth are: \href{https://www.cs.rice.edu/~johnmc/comp522/lecture-notes/COMP522-2019-Lecture3-Multithreading.pdf}{SUN Niagara 2}, \href{https://ieeexplore.ieee.org/document/7477467}{Knights landing (KNL): 2nd Generation Intel Xeon Phi processor}.