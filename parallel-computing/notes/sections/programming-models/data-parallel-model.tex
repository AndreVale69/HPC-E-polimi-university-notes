\subsection{Data-Parallel model}

In parallel computing, the \definition{Data-Parallel model} is characterized by \textbf{applying the same operation simultaneously across multiple data points}. This model is particularly effective for tasks involving large datasets where the same computation needs to be performed on each data element.

\highspace
This model organizes \textbf{computation as operations on sequences of elements} (e.g., perform the same function on all elements of a sequence). In programming languages, the basic data type used for this purpose is called \texttt{Sequence} (e.g., C++). Despite the name of the datatype used in the languages, the definition is that it \textbf{is an ordered collection of elements, where each element can be accessed and manipulated using various sequence operators}. The most common operators are:
\begin{itemize}
    \item \textbf{\underline{Map}}. The \texttt{map} function is a \textbf{higher-order function}\footnote{%
        A \definitionWithSpecificIndex{higher-order function}{Higher-Order function}{} is a function that can do one or both of the following:
        \begin{itemize}
            \item Take other functions as arguments (parameters).
            \item Returns a function or value as its result.
        \end{itemize}
    } that \textbf{operates on sequences}. It applies a \textbf{side-effect-free unary function}\footnote{%
        A function that takes only one argument and doesn't suffer from side effects.
    } $f : a \rightarrow b$ \textbf{to all elements of an input sequence}, producing an \textbf{output sequence of the same length}.
    \begin{examplebox}[: Python Analogy]
        For a better understanding, we provide a very simple Python code to see how a \texttt{map} function works. In Python there is a \href{https://docs.python.org/3/library/functions.html#map}{\texttt{map}} function that does exactly what we say.
        \begin{lstlisting}[language=Python]
# Define a trivial function that squares a number
def square(x):
    return x * x

# Create a sample list
numbers = [1, 2, 3, 4, 5]

# Use the map function to apply the 'square' function
# to each element in the 'numbers' list
squared_numbers = map(square, numbers)

# Convert the result to a list and print it
print(list(squared_numbers))\end{lstlisting}
        And the result is:
        \begin{lstlisting}
[1, 4, 9, 16, 25]\end{lstlisting}
    \end{examplebox}

    Now the main idea is: since the \texttt{map} is a function without side effects, we can \textbf{apply it to all elements of the sequence in any order without changing the output of the program}. This allows \textbf{reordering or parallel processing of sequence elements to \textcolor{Green4}{\textbf{optimize performance}}}.
    
    \item \textbf{\underline{Reduction}}. The reduction born from the need to make parallel operations of iterations. For example, in a \texttt{for} loop, if we need a progressive sum of an element, we should implement some kind of mechanism to manage the synchronization. Using the reduction strategies, the compiler will do this job. We suggest reading the Reduction section (\ref{paragraph: Reduction}, page \pageref{paragraph: Reduction}) in the OpenMP chapter to understand what we mean.
    
    \item Others like \textbf{\underline{scan}} and \textbf{\underline{shift}}.
\end{itemize}