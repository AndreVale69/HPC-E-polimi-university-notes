\subsubsection{GPGPU Best Practices}

NVIDIA provides a guide to help developers get the most out of NVIDIA CUDA GPUs:
\begin{center}
    \href{https://docs.nvidia.com/cuda/cuda-c-best-practices-guide/}{CUDA C++ Best Practices Guide} \hspace{2em} \qrcode{https://docs.nvidia.com/cuda/cuda-c-best-practices-guide/}
\end{center}

\noindent
Among the suggestions, NVIDIA explains a process for accelerating an application with NVIDIA GPUs called the \emph{Cyclical Process for Accelerating Applications with NVIDIA GPUs}.

\highspace
The \definition{Cyclical Process for Accelerating Applications with NVIDIA GPU}s is a \textbf{systematic approach designed to optimize and enhance the performance of applications by leveraging the computational power of NVIDIA GPUs}. This process ensures that applications can take full advantage of the parallel processing capabilities of GPUs for improved performance and efficiency. The process consists of four main stages: Assess, Parallelize, Optimize, and Deploy.
\begin{enumerate}
    \item \important{Assess}.
    \begin{enumerate}
        \item \textbf{Locate Bottlenecks}. Identify the parts of our application where performance is limited. This involves profiling our application to understand where most of the computation time is spent.

        \item \textbf{Estimate Parallelization Benefits}. Determine the potential performance improvements from parallelizing the application. Evaluate how much of the workload can be efficiently offloaded to the GPU.
    \end{enumerate}

    \item \important{Parallelize}.
    \begin{enumerate}
        \item \textbf{Apply Libraries, Compiler Directives, or CUDA}. Utilize GPU-optimized libraries (such as cuBLAS for linear algebra operations), apply compiler directives (like OpenACC) to guide the parallelization, or write custom CUDA kernels to parallelize the identified computational bottlenecks.
    \end{enumerate}

    \item \important{Optimize}.
    \begin{enumerate}
        \item \textbf{Apply Optimizations}. Fine-tune our GPU code to improve performance. This can include optimizing memory access patterns to reduce latency, ensuring efficient data transfer between the CPU and GPU, and maximizing thread utilization to fully exploit the GPU's parallel architecture.
        
        \item \textbf{Measure Performance Improvements}. Continuously profile and benchmark our application to monitor the impact of the optimizations and ensure that they lead to significant performance gains.
    \end{enumerate}
    
    \newpage

    \item \important{Deploy}.
    \begin{enumerate}
        \item \textbf{Compare with Performance Estimations}. Verify that the performance improvements achieved through parallelization and optimization meet the initial estimations. This step ensures that the application performs as expected in a real-world environment.

        \item \textbf{Move to Production}. Once satisfied with the performance, deploy the optimized application to production. This involves integrating the GPU-accelerated code into the main application and ensuring it runs efficiently in the production environment.
    \end{enumerate}
\end{enumerate}
The process is depicted as a \emph{continuous cycle}, indicating that \textbf{optimization and assessment are ongoing activities}. This \textbf{iterative approach ensures that the application remains optimized and continues to benefit from GPU acceleration over time}.

\highspace
By following this cyclical process, developers can systematically \textbf{identify performance bottlenecks}, \textbf{efficiently parallelize} and \textbf{optimize} their \textbf{applications}, and \textbf{achieve significant performance improvements using NVIDIA GPUs}. This methodical approach helps ensure that the application takes full advantage of GPU capabilities, leading to faster, more efficient computing.