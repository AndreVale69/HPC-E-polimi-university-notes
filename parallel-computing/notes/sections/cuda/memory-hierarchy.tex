\subsection{Memory hierarchy}

The memory hierarchy in CUDA was explained in Section \ref{subsubsection: Memory model}, page \pageref{subsubsection: Memory model}. Here is a small summary:
\begin{itemize}
    \item \important{Per-thread} (registers):
    \begin{itemize}
        \item \textbf{Scope}: Per thread.
        \item \textbf{Lifetime}: Duration of the thread.
        \item \textbf{Speed}: Fastest memory in the CUDA memory hierarchy.
        \item \textbf{Usage}: Used for storing frequently accessed variables and temporary data.
        \item \textbf{Limitations}: Limited in size; excessive usage can lead to spilling into local memory, which is slower.
    \end{itemize}

    \item \important{Per-block} (shared memory, cache):
    \begin{itemize}
        \item \textbf{Scope}: Per block.
        \item \textbf{Lifetime}: Duration of the block.
        \item \textbf{Speed}: Much faster than global memory, similar to L1 cache.
        \item \textbf{Usage}: Used for data sharing among threads within the same block, enabling efficient inter-thread communication.
        \item \textbf{Configuration}: Typically user-managed, allowing for explicit control over data placement.
        \item \textbf{Limitations}: Limited in size; excessive usage can limit the number of active blocks per SM.
    \end{itemize}

    \item \important{Global Memory} (off-chip DRAM):
    \begin{itemize}
        \item \textbf{Scope}: All threads.
        \item \textbf{Lifetime}: Duration of the application.
        \item \textbf{Speed}: Significantly slower than shared memory and registers due to higher latency.
        \item \textbf{Usage}: Used for data that needs to be accessed by multiple threads or blocks.
        \item \textbf{Characteristics}: Large in size, but high latency; optimizing access patterns (coalesced accesses) can improve performance.
    \end{itemize}
\end{itemize}

\newpage

\begin{flushleft}
    \textcolor{Green3}{\faIcon{book} \textbf{Memory Accessibility}}
\end{flushleft}
\begin{itemize}
    \item \important{Per-thread}:
    \begin{itemize}
        \item \textbf{Host}: cannot directly access registers.
        \item \textbf{Device}: registers are accessed exclusively by the thread that owns them.
        \item \textbf{Operations}:
        \begin{itemize}
            \item \textbf{Host}: none. The host cannot allocate, deallocate, or access registers directly.
            \item \textbf{Device}: each thread can read from and write to its own registers. But registers are private to the thread, meaning \emph{no other thread can access another thread's registers}.
        \end{itemize}
    \end{itemize}

    \item \important{Per-block}:
    \begin{itemize}
        \item \textbf{Host}: cannot directly access shared memory, but can specify shared memory size for each kernel launch.
        \item \textbf{Device}: can read from and write to shared memory within a block.
        \item \textbf{Operations}:
        \begin{itemize}
            \item \textbf{Host}: none directly (can specify size in kernel launch).
            \item \textbf{Device}: Read/Write.
        \end{itemize}
    \end{itemize}

    \item \important{Global Memory}:
    \begin{itemize}
        \item \textbf{Host}: can allocate and deallocate memory using \texttt{cudaMalloc} and \texttt{cudaFree}. Can copy data to/from global memory using \texttt{cudaMemcpy}.
        \item \textbf{Device}: can read from and write to global memory directly.
        \item \textbf{Operations}:
        \begin{itemize}
            \item \textbf{Host}: \texttt{cudaMalloc}, \texttt{cudaFree}, \texttt{cudaMemcpy}, \texttt{cudaMemset}.
            \item \textbf{Device}: Read/Write.
        \end{itemize}
    \end{itemize}
\end{itemize}

\newpage

\begin{flushleft}
    \textcolor{Green3}{\faIcon{memory} \textbf{Memory Management APIs}}
\end{flushleft}
\begin{itemize}
    \item \important{\texttt{cudaMalloc}}:
    \begin{itemize}
        \item \textbf{Purpose}: allocates memory on the GPU.
        \item \textbf{Usage}: used when we need to allocate space for variables or arrays that the GPU will use.
        \item \textbf{Function signature}:
        \begin{lstlisting}[language=C++]
cudaError_t cudaMalloc(void** devPtr, size_t size);\end{lstlisting}
        \item \textbf{Parameters}:
        \begin{itemize}
            \item \texttt{devPtr}: pointer to the allocated device memory.
            \item \texttt{size}: size in bytes of the allocated memory.
        \end{itemize}
        \item \example{Example}:
        \begin{lstlisting}[language=C++]
float* d_array;
cudaMalloc((void**)&d_array, N * sizeof(float));\end{lstlisting}
    \end{itemize}

    \item\important{\texttt{cudaMallocHost}}:
    \begin{itemize}
        \item \textbf{Purpose}: allocates pinned memory on the host.
        \item \textbf{Usage}: used for host memory that can be asynchronously copied to the device, improving transfer efficiency.
        \item \textbf{Function signature}:
        \begin{lstlisting}[language=C++]
cudaError_t cudaMallocHost(void** ptr, size_t size);\end{lstlisting}
        \item \textbf{Parameters}:
        \begin{itemize}
            \item \texttt{ptr}: pointer to the allocated host memory.
            \item \texttt{size}: size in bytes of the allocated memory.
        \end{itemize}
        \item \example{Example}:
        \begin{lstlisting}[language=C++]
float* h_array;
cudaMallocHost((void**)&h_array, N * sizeof(float));\end{lstlisting}
    \end{itemize}

    \item \important{\texttt{cudaFree}}:
    \begin{itemize}
        \item \textbf{Purpose}: frees memory that was allocated on the GPU.
        \item \textbf{Usage}: used to deallocate memory that was previously allocated with \texttt{cudaMalloc}.
        \item \textbf{Function signature}:
        \begin{lstlisting}[language=C++]
cudaError_t cudaFree(void* devPtr);\end{lstlisting}
        \item \textbf{Parameters}:
        \begin{itemize}
            \item \texttt{devPtr}: pointer to the memory to be freed.
        \end{itemize}
        \item \example{Example}:
        \begin{lstlisting}[language=C++]
float* d_array;
cudaFree(d_array);\end{lstlisting}
    \end{itemize}

    \item \important{\texttt{cudaFreeHost}}:
    \begin{itemize}
        \item \textbf{Purpose}: frees pinned memory that was allocated on the host.
        \item \textbf{Usage}: used to deallocate memory that was previously allocated with \texttt{cudaMallocHost}.
        \item \textbf{Function signature}:
        \begin{lstlisting}[language=C++]
cudaError_t cudaFreeHost(void* ptr);\end{lstlisting}
        \item \textbf{Parameters}:
        \begin{itemize}
            \item \texttt{ptr}: pointer to the memory to be freed.
        \end{itemize}
        \item \example{Example}:
        \begin{lstlisting}[language=C++]
float* h_array;
cudaFreeHost(h_array);\end{lstlisting}
    \end{itemize}

    \item \important{\texttt{cudaMemcpy}}:
    \begin{itemize}
        \item \textbf{Purpose}: copies data between host and device, or between different regions of device memory.
        \item \textbf{Usage}: used for transferring data to and from the GPU, or between different memory regions on the GPU.
        \item \textbf{Function signature}:
        \begin{lstlisting}[language=C++]
cudaError_t cudaMemcpy(void* dst, const void* src, size_t count, cudaMemcpyKind kind);\end{lstlisting}
        \item \textbf{Parameters}:
        \begin{itemize}
            \item \texttt{dst}: destination pointer.
            \item \texttt{src}: source pointer.
            \item \texttt{count}: size in bytes of the memory to be copied.
            \item \texttt{kind}: type of copy operation (e.g., \texttt{cudaMemcpyHostToDevice}, \texttt{cudaMemcpyDeviceToHost}, \texttt{cudaMemcpyDeviceToDevice})
        \end{itemize}
        \item \example{Example}:
        \begin{lstlisting}[language=C++]
cudaMemcpy(d_array, h_array, N * sizeof(float), cudaMemcpyHostToDevice);\end{lstlisting}
    \end{itemize}
\end{itemize}

\begin{flushleft}
    \textcolor{Red2}{\faIcon{exclamation-triangle} \textbf{\texttt{cudaMallocHost}/\texttt{cudaFreeHost} can reduce CPU performance}}
\end{flushleft}
\definition{Pinned memory} (or page-locked memory) is a \textbf{region of system memory that is \dquotes{locked} and cannot be paged out by the operating system}. This type of memory provides faster data transfer rates between the CPU and GPU because it allows for direct memory access (DMA) without the need for the CPU to perform intermediate steps. Since \texttt{cudaMallocHost} and \texttt{cudaFreeHost} work on this type of memory, the possible worst-case scenarios are:
\begin{itemize}
    \item[\textcolor{Red2}{\faIcon{times}}] \important{Limited System Resources}. \textbf{Pinned memory consumes physical RAM}. When we allocate a large amount of pinned memory, it reduces the available RAM for other system tasks, which can lead to increased memory pressure and reduced performance for other applications.


    \item[\textcolor{Red2}{\faIcon{times}}] \important{Increased Memory Allocation Time}. \textbf{Allocating pinned memory} (\texttt{cudaMallocHost}) \textbf{is generally more time-consuming} than allocating pageable memory because the operating \textbf{system must ensure that the allocated memory pages are locked in RAM and cannot be paged out}.
    
    Similarly, freeing pinned memory (\texttt{cudaFreeHost}) involves \textbf{unlocking these pages, which can also be a time-consuming operation}.


    \item[\textcolor{Red2}{\faIcon{times}}] \important{Memory Fragmentation}. \textbf{Frequent allocations and deallocations of pinned memory can lead to memory fragmentation}. Over time, this fragmentation can make it harder for the operating system to find contiguous blocks of free memory, which can slow down memory allocation and deallocation operations.


    \item[\textcolor{Red2}{\faIcon{times}}] \important{Reduced Cache Efficiency}. Pinned memory allocations can affect the CPU's cache efficiency. \textbf{When a large portion of memory is pinned, it may reduce the effectiveness of the CPU's memory caching mechanisms}, leading to increased memory access times for other processes.
\end{itemize}
Is suggest to use the pinned memory when:
\begin{itemize}
    \item[\textcolor{Green3}{\faIcon{check}}] \textcolor{Green3}{\textbf{Performance-Critical Data Transfers}}. Use pinned memory for data transfers that are critical for performance, where the speedup from faster data transfer rates outweighs the potential downsides.
    
    \item[\textcolor{Green3}{\faIcon{check}}] \textcolor{Green3}{\textbf{Async Operations}}. Pinned memory is particularly beneficial for asynchronous data transfers between the host and device, enabling overlap of computation and data transfer for better performance.
\end{itemize}

\begin{examplebox}[: Manual Memory Management - Vector Addition]
    \begin{lstlisting}[language=C++]
% allocate h_A, h_B, h_C
void vecAdd(float *h_A, float *h_B, float *h_C, int n) {
    int size = n * sizeof(float);
    float *d_A, *d_B, *d_C;

    // Allocate memory on the GPU (device)
    // for arrays d_A, d_B, d_C
    cudaMalloc((void**)&d_A, size);
    cudaMalloc((void**)&d_B, size);
    cudaMalloc((void**)&d_C, size);

    // Copy data from host arrays h_A and h_B
    // to device arrays d_A and d_B
    cudaMemcpy(d_A, h_A, size, cudaMemcpyHostToDevice);
    cudaMemcpy(d_B, h_B, size, cudaMemcpyHostToDevice);

    // Kernel invocation code (not shown in the image)
    // Example kernel launch:
    // vecAddKernel<<<blocks, threads>>>(d_A, d_B, d_C, n);

    // Copy the result from device array
    // d_C back to host array h_C
    cudaMemcpy(h_C, d_C, size, cudaMemcpyDeviceToHost);

    // Free the allocated memory on the device
    cudaFree(d_A);
    cudaFree(d_B);
    cudaFree(d_C);
}
% free h_A, h_B, h_C\end{lstlisting}
    
    \begin{itemize}
        \item \textbf{Allocate Host Memory}. Before the \texttt{vecAdd} function is called, host memory for arrays \texttt{h\_A}, \texttt{h\_B}, and \texttt{h\_C} is allocated. This memory is used to store the input data and the result on the host (CPU).

        \item \textbf{Function Definition}. The \texttt{vecAdd} function performs the steps in the code.

        \item \textbf{Memory Allocation on Device}. \texttt{cudaMalloc} allocates memory on the GPU for \texttt{d\_A}, \texttt{d\_B}, and \texttt{d\_C}. The size of each array is determined by \texttt{n * sizeof(float)}.
        
        \item \textbf{Copy Data from Host to Device}. \texttt{cudaMemcpy} copies data from the host arrays \texttt{h\_A} and \texttt{h\_B} to the device arrays \texttt{d\_A} and \texttt{d\_B} using the \texttt{cudaMemcpyHostToDevice} flag. This step transfers the input data from the CPU to the GPU for processing.

        \item \textbf{Kernel Invocation}. The CUDA kernel (not shown in the image) is invoked to perform vector addition on the device. Each thread on the GPU computes one element of the result vector.

        \item \textbf{Copy Data from Device to Host}. \texttt{cudaMemcpy} copies the result from the device array \texttt{d\_C} back to the host array \texttt{h\_C} using the \texttt{cudaMemcpyDeviceToHost} flag. This step transfers the computed result from the GPU to the CPU.

        \item \textbf{Free Allocated Memory}. \texttt{cudaFree} frees the allocated memory on the GPU for \texttt{d\_A}, \texttt{d\_B}, and \texttt{d\_C}. This ensures that no memory leaks occur on the GPU.
    \end{itemize}
\end{examplebox}

\begin{flushleft}
    \textcolor{Green3}{\faIcon{question-circle} \textbf{How can we allocate memory that is shared between CPU and GPU?}}
\end{flushleft}
\definition{Unified Memory in CUDA} is a \textbf{single memory space that is accessible by both the host (CPU) and the device (GPU)}. It abstracts away the complexities of explicit memory management, making it easier to develop CUDA applications. Its features:
\begin{itemize}
    \item \important{Single Address Space}. Both the \textbf{CPU and GPU can access the same memory location}, \emph{eliminating the need for explicit data transfers between them}. This means that the same pointer can be used on both the host and the device.

    \item \important{Automatic Data Migration}. The CUDA runtime system handles the movement of data between the host and device automatically. When the GPU accesses data that resides in the host memory, the CUDA runtime migrates the data to the GPU memory and vice versa.

    \item \important{On-Demand (Page) Migration}. Unified Memory uses a demand-\break paging mechanism similar to virtual memory in operating systems. \textbf{When a page of memory is accessed by the CPU or GPU, and it is not currently resident in the respective memory, it is migrated on demand}.

    \item[\textcolor{Green3}{\faIcon{check-circle}}] \textcolor{Green3}{\textbf{Pros}}
    \begin{itemize}
        \item[\textcolor{Green3}{\faIcon{check}}] \textcolor{Green3}{\textbf{Simplified Memory Management}}. Developers do not need to worry about explicit memory copies (\texttt{cudaMemcpy}). The \textbf{CUDA runtime manages data movement automatically}.
        
        \item[\textcolor{Green3}{\faIcon{check}}] \textcolor{Green3}{\textbf{Ease of Programming}}. Writing CUDA programs becomes easier, especially for those new to CUDA, as \textbf{the same pointer can be used for both host and device operations}.
        
        \item[\textcolor{Green3}{\faIcon{check}}] \textcolor{Green3}{\textbf{Efficient Use of Memory}}. Reduces code complexity, as there is \textbf{no need for separate host and device memory allocations and management}.
    \end{itemize}

    \item[\textcolor{Red2}{\faIcon{times-circle}}] \textcolor{Red2}{\textbf{Cons}}
    \begin{itemize}
        \item[\textcolor{Red2}{\faIcon{times}}] \textcolor{Red2}{\textbf{Potential Performance Overhead}}. The \textbf{automatic data migration can introduce runtime overhead due to page faults}\footnote{
            Modern operating systems use a virtual memory system where the physical memory (RAM) is abstracted as a larger virtual memory space. This allows programs to use more memory than physically available by using disk storage (paging). A \definitionWithSpecificIndex{page fault}{Page Fault}{} occurs when \textbf{a program tries to access a part of the memory that is not currently in the physical memory (RAM)}. The operating system intervenes to load the required memory page from disk into RAM, allowing the program to continue execution.
        } when data is accessed. 
        
        Unified Memory uses a similar (to operating system) paging mechanism to manage memory across the CPU and GPU. When a memory page is accessed by the CPU or GPU and is not present in the respective memory (host or device), a page fault occurs. The CUDA runtime automatically migrates the page from the current memory location (host to device or device to host) to the memory where the access occurred.

        This can be less efficient compared to explicitly managed memory transfers.
        
        \item[\textcolor{Red2}{\faIcon{times}}] \textcolor{Red2}{\textbf{Limited Control}}. Developers have less control over when and where data is moved. For performance-critical applications, \textbf{manual memory management can sometimes yield better performance}.
        
        \item[\textcolor{Red2}{\faIcon{times}}] \textcolor{Red2}{\textbf{Memory Contention}}. Sharing the same memory space between the CPU and GPU can lead to contention, \textbf{potentially impacting performance if both are trying to access the same memory simultaneously}.
    \end{itemize}
\end{itemize}

\newpage

\noindent
\texttt{cudaMallocManaged} is a function in CUDA that \textbf{allocates managed memory}, \textbf{which is accessible by both the host (CPU) and the device (GPU)}. This unified memory simplifies the programming model by allowing both the CPU and GPU to share a single memory space.
\begin{lstlisting}[language=C++]
cudaError_t cudaMallocManaged(void **devPtr, size_t size);  
\end{lstlisting}
\begin{itemize}
    \item \texttt{devPtr}: pointer to the allocated unified memory.
    \item \texttt{size}: size in bytes of the allocated memory.
\end{itemize}

\begin{flushleft}
    \textcolor{Green3}{\faIcon{check-circle} \textbf{How to avoid page faults with \texttt{cudaMallocManaged}}}
\end{flushleft}
\definition{Asynchronous memory prefetching} is a \textbf{technique used to proactively move data between the host and device to avoid page faults} and improve performance. This is particularly useful in Unified Memory, where data can be accessed by both the CPU and GPU.

\highspace
Instead of waiting for a page fault to occur (which would trigger an on-demand data transfer, a scenario we would avoid due to performance overhead), we prefetch the required data to the desired memory location before it is needed. This \textbf{operation is done asynchronously}, meaning that it \textbf{does not block the execution of other operations}. This allows the program to continue running while the data is being transferred in the background.

\highspace
In CUDA, this technique is used with the function \texttt{cudaMemPrefetchAsync}:
\begin{lstlisting}[language=C++]
cudaError_t cudaMemPrefetchAsync(
    void *devPtr, size_t count, int dstDevice, cudaStream_t stream = 0
);
\end{lstlisting}
\begin{itemize}
    \item \texttt{devPtr}: pointer to the memory to be prefetched.
    \item \texttt{count}: size of the memory to be prefetched, in bytes.
    \item \texttt{dstDevice}: the destination device (can be a GPU device ID or the result of the function \texttt{cudaCpuDeviceId} for the host).
    \item \texttt{stream}: the stream to perform the operation in (optional).
\end{itemize}
An \example{example} of use:
\begin{lstlisting}[language=C++]
int deviceId;
cudaGetDevice(&deviceId);

// Prefetch data to the device
cudaMemPrefetchAsync(pointerToSomeUMData, size, deviceId);

// Prefetch data back to the host
cudaMemPrefetchAsync(pointerToSomeUMData, size, cudaCpuDeviceId);
\end{lstlisting}