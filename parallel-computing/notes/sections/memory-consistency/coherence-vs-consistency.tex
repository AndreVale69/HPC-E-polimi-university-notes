\section{Memory Consistency}

\subsection{Coherence vs Consistency}

\begin{flushleft}
    \textcolor{Green3}{\faIcon{question-circle} \textbf{Why Memory Consistency?}}
\end{flushleft}
Memory consistency models are fundamental to parallel computing because they define the rules that govern the visibility and ordering of memory operations across multiple threads. Unlike sequential computing, where the order of operations is straightforward, parallel computing introduces complexity due to concurrent execution. This concurrency can lead to race conditions, making it difficult to determine the \dquotes{latest} value of a shared memory location.

\highspace
\begin{flushleft}
    \textcolor{Green3}{\faIcon{question-circle} \textbf{Why Memory Models are Fundamental to Parallel Computing}}
\end{flushleft}
In an ideal scenario, loads (reads) should return the \emph{most recent} value written to a memory location. However, defining and achieving this \dquotes{most recent} value in a parallel environment is \textbf{complicated because multiple threads may be interacting with the same memory at the same time}. This is where memory consistency models play a critical role.

\highspace
When we talk about memory, there are two fundamental concepts to explore: coherence and consistency.
\begin{itemize}
    \item \definition{Memory Coherence} ensures that all processors see a consistent view of a single memory location.
    
    \textbf{It defines the requirements for the observed behavior of reads and writes to the \underline{same} memory location}:
    \begin{itemize}
        \item \textbf{All processors must agree on the order of reads/writes to a single memory location (\texttt{X})}. This means that if one processor writes a value to a memory location, all other processors should eventually see that value.
        
        \item \textbf{A timeline of operations involving the same memory location}. It is possible to create a timeline such that the observations of all processors are consistent with that timeline. This ensures that each processor sees the most recent write to a memory location in the correct order.
    \end{itemize}


    \item \definition{Memory Consistency} extends this concept to the entire memory, ensuring an apparent ordering of operations, which dictates how memory operations performed by one thread become visible to other threads.
    
    \textbf{It defines the behavior of reads/writes to \underline{different} memory locations}:
    \begin{itemize}
        \item \textbf{Coherence guarantees eventual propagation}. Coherence ensures that writes to a single memory location (\texttt{X}) will eventually propagate to other processors.
        
        \item \textbf{Consistency deals with the timing of propagation}. Consistency addresses when writes to one memory location (\texttt{X}) propagate to other processors relative to reads and writes to other memory locations. This means that consistency ensures a predictable order for operations across multiple memory locations, considering their interactions.
    \end{itemize}
\end{itemize}

\begin{examplebox}[: Chronology of Operations]
    Imagine the following chronology of operations on a memory address (\texttt{X}):
    \begin{enumerate}[label={$t = $ \arabic*.}]
        \item Write. \texttt{P0} $\xlongrightarrow{5}$ \texttt{X}
        \item Read. \texttt{P1} $\xlongleftarrow{5}$ \texttt{X}
        \item Write. \texttt{P2} $\xlongrightarrow{10}$ \texttt{X}
        \item Write. \texttt{P2} $\xlongrightarrow{11}$ \texttt{X}
        \item Read. \texttt{P1} $\xlongleftarrow{11}$ \texttt{X}
    \end{enumerate}
    This example demonstrates how:
    \begin{itemize}
        \item Memory \textbf{\emph{coherence}} ensures that \textbf{all processors} eventually \textbf{agree on the order of operations for the same memory location}.

        \item Memory \textbf{\emph{consistency}} ensures a \textbf{predictable interaction between operations on different memory locations}.
    \end{itemize}
\end{examplebox}

\highspace
\begin{flushleft}
    \textcolor{Green3}{\faIcon{question-circle} \textbf{What happens if there is a cache system?}}
\end{flushleft}
Modern parallel computing systems rely heavily on caches to improve performance. Therefore, \textbf{ensuring that all caches maintain a consistent view of memory} (cache coherence) \textbf{becomes critical to the proper operation of the system}. While memory coherence is about maintaining a consistent view of memory, cache coherence is about implementing this consistency in systems where each processor has its own local cache. So here we define again the concepts of consistency and coherence in the cache environment.
\begin{itemize}
    \item \definition{Cache Coherence}. This is a more specific implementation of memory coherence. In general, caches are used to store copies of frequently accessed data to speed up processing. However, when multiple processors modify their cached copies of the same memory location, inconsistencies can occur.
    
    Therefore, the main goal is to ensure that the memory system in a parallel computer behaves as if the caches were not there. This is similar to how a memory system in a uni-processor system behaves as if the cache were not there.

    In a system without caches, there would be no need for cache coherence. Cache coherence \textbf{ensures that all processors see a consistent view of memory, even though each processor may have a local cache}. This means that any changes made to a memory location by one processor will eventually be reflected in the caches of the other processors.

    
    \item \definition{Memory Consistency} (in the cache environment). Defines the allowed behavior of loads (reads) and stores (writes) to different addresses in a parallel system. This behavior should be specified whether or not caches are present. A memory consistency model specifies the rules for the order in which memory operations (loads and stores) become visible to other threads. It ensures that all processors in the system observe memory operations in a predictable and coherent manner, regardless of the presence of caches.

    In other words, it is the same definition presented on the previous page, but we emphasize that model should also work with the cache.
\end{itemize}
