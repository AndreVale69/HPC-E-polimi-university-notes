\subsection{Basic syntax}

We can manage OpenMP work flow using the directive syntax. We remember that the reference guide of OpenMP is available on their website:
\begin{center}
    \href{https://www.openmp.org/resources/refguides/}{Reference Guide} \hspace{2em} \qrcode{https://www.openmp.org/resources/refguides/}
\end{center}

\highspace
A \textbf{directive} is a combination of the base-language mechanism and a \emph{directive-specification} (the \emph{directive-name} followed by \emph{optional clauses}). A construct consists of a directive and, often, additional base language code. In C++ directives are formed from either pragmas or attributes.
\begin{openmpbox}[: \texttt{pragma omp}]
\begin{lstlisting}[language=C++, mathescape=true]
#pragma omp $\texttt{\emph{directive-specification}}$
\end{lstlisting}
\end{openmpbox}

\noindent
The \textbf{number of OpenMP threads can be set} using:
\begin{itemize}
    \item At compilation time: using the environment variable \texttt{OMP\_NUM\_THREADS}
    \item At runtime: using the function
    \marginpar{
        \href{https://www.openmp.org/spec-html/5.0/openmpsu110.html\#x147-6380003.2.1} {Doc. \faIcon{book}}
    }
    \begin{openmpbox}[: \texttt{omp\_set\_num\_threads}]
        \begin{lstlisting}[language=C++]
void omp_set_num_threads(int num_threads)\end{lstlisting}
    \end{openmpbox}
\end{itemize}
Other useful function to get information about threads:
\begin{itemize}
    \item The \textbf{number of threads in the current team}:
    \marginpar{
        \href{https://www.openmp.org/spec-html/5.0/openmpsu111.html\#x148-6450003.2.2} {Doc. \faIcon{book}}
    }
    \begin{openmpbox}[: \texttt{omp\_get\_num\_threads}]
        \begin{lstlisting}[language=C++]
int omp_get_num_threads()\end{lstlisting}
    \end{openmpbox}
    The binding region for an \texttt{omp\_get\_num\_threads} region is the innermost enclosing \texttt{parallel} region. If called from the sequential part of a program, this routine returns 1.

    \newpage

    \item The \textbf{upper bound on the number of threads} that could be used to form a new team if a \texttt{parallel} construct without a \texttt{num\_threads} clause were encountered after execution returns from this routine.
    \marginpar{
        \href{https://www.openmp.org/spec-html/5.0/openmpsu112.html\#x149-6510003.2.3} {Doc. \faIcon{book}}
    }
    \begin{openmpbox}[: \texttt{omp\_get\_max\_threads}]
        \begin{lstlisting}[language=C++]
int omp_get_max_threads()\end{lstlisting}
    \end{openmpbox}

    \item The \textbf{thread number of the calling thread}, within the current team.
    \marginpar{
        \href{https://www.openmp.org/spec-html/5.0/openmpsu113.html\#x150-6570003.2.4} {Doc. \faIcon{book}}
    }
    \begin{openmpbox}[: \texttt{omp\_get\_thread\_num}]
        \begin{lstlisting}[language=C++]
int omp_get_thread_num()\end{lstlisting}
    \end{openmpbox}
    
    \item The \textbf{elapsed wall clock time in seconds}.
    \marginpar{
        \href{https://www.openmp.org/spec-html/5.0/openmpsu160.html\#x199-9660003.4.1} {Doc. \faIcon{book}}
    }
    \begin{openmpbox}[: \texttt{omp\_get\_wtime}]
    	\begin{lstlisting}[language=C++]
double omp_get_wtime()\end{lstlisting}
    \end{openmpbox}
    
    \item The \textbf{precision of the timer (seconds between ticks)} used by\break \texttt{omp\_get\_wtick}.
    \marginpar{
        \href{https://www.openmp.org/spec-html/5.0/openmpsu161.html\#x200-9710003.4.2} {Doc. \faIcon{book}}
    }
    \begin{openmpbox}[: \texttt{omp\_get\_wtick}]
    	\begin{lstlisting}[language=C++]
double omp_get_wtick()\end{lstlisting}
    \end{openmpbox}
\end{itemize}

\highspace
OpenMP programs execute serially until they reach a \texttt{parallel} directive. As we have explained at page \pageref{figure: how OpenMP works}, the thread that was executing the code spawns a group of \dquotes{slave} threads and becomes the \dquotes{master} (thread ID 0). The code in the structured block is replicated, each thread executes a copy. At the end of the block there is an implied barrier, only the \dquotes{master} thread continues.
\begin{openmpbox}[: \texttt{pragma omp parallel}]
\begin{lstlisting}[language=C++, mathescape=true]
#pragma omp parallel $\texttt{\emph{optional-clauses}}$\end{lstlisting}
\end{openmpbox}

\noindent
The parallel directive has \textbf{optional} clauses, the most commonly used are:
\begin{itemize}
    \item Specify the \textbf{number of threads to spawn}:
    \begin{lstlisting}[language=C++]
#pragma omp parallel num_threads(int)\end{lstlisting}

	\newpage

    \item Conditional parallelization with:
\begin{lstlisting}[language=C++]
#pragma omp parallel if (condition)\end{lstlisting}
    \begin{examplebox}[: parallel if condition]
\begin{lstlisting}[language=C++]
#include <stdio.h>
#include <omp.h>

void test(int val)
{
    #pragma omp parallel if (val != 0)
    if (omp_in_parallel()) {
        #pragma omp single
        printf_s(
            "val = %d, parallelized with %d threads\n",
            val, omp_get_num_threads()
        );
    } else {
        printf_s("val = %d, serialized\n", val);
    }
}

int main( )
{
    omp_set_num_threads(2);
    test(0);
    test(2);
}\end{lstlisting}
    The output will be:
\begin{lstlisting}
val = 0, serialized
val = 2, parallelized with 2 threads\end{lstlisting}
    \end{examplebox}

    \item Data scope clauses (explained in the following pages).
\end{itemize}
The \textbf{number of threads in a parallel region is determined by the following factors}, in order of priority (high to low):
\begin{enumerate}
    \item Evaluation of the \texttt{if} clause;
    \item Value of the \texttt{num\_threads} clause;
    \item Use of the \texttt{omp\_set\_num\_threads()} library function;
    \item Setting of the \texttt{OMP\_NUM\_THREADS} environment variable;
    \item Implementation default, e.g., the number of CPUs on a node.
\end{enumerate}
