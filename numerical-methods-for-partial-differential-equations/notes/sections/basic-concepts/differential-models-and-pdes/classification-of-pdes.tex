\subsubsection{Classification of PDEs}

A PDE is a relationship among:
\begin{itemize}
    \item The partial derivatives of a function $u = u\left(\mathbf{u}, t\right)$, that is the PDE \textbf{solution};
    \item \textbf{Spatial coordinates} $\mathbf{x} = \left(x_{1}, \dots, x_{d}\right)^{T} \in \mathbb{R}^{d}$ on which the solution depends (if the problem is defined in a spatial domain $\Omega \subset \mathbb{R}^{d}$).
    \item \textbf{Time variable} $t$.
\end{itemize}
Therefore, a PDE can be written as:
\begin{equation}
    \mathcal{P}\left(
        u,
        \dfrac{\partial u}{\partial t},
        \dfrac{\partial u}{\partial x_{1}},
        \dots,
        \dfrac{\partial u}{\partial x_{d}},
        \dots,
        \dfrac{\partial^{p_{1} + \dots + p_{d} + p_{t}} u}{\partial x_{1}^{p_{1}} \dots \partial x_{d}^{p_{d}} \: \partial t^{p_{t}}},
        \mathbf{x},
        t;
        g
    \right) = 0
\end{equation}
Where $p_{1}, \dots, p_{d}, p_{t} \in \mathbb{N}$ and $g$ are the data.

\highspace
\begin{definitionbox}[: PDE order]
    The \definition{PDE order} is the \textbf{maximum order of derivation} that appears in $\mathcal{P}$, that is:
    \begin{equation}
        q = p_{1} + \dots + p_{d} + p_{t}
    \end{equation}
\end{definitionbox}

\begin{definitionbox}[: PDE is linear]
    The \definition{PDE is linear} if $\mathcal{P}$ \textbf{linearly depends} on $u$ and its \textbf{derivatives}.
\end{definitionbox}

\highspace
\begin{flushleft}
    \textcolor{Green3}{\faIcon{square-root-alt} \textbf{Classification}}
\end{flushleft}
Let us focus on linear PDEs of order $q = 2$ with constant coefficients, so that the general PDE formulation is:
\begin{equation*}
    \mathcal{L}u = g
\end{equation*}
Where $\mathcal{L}$ is a second order, \textbf{linear differential operator}. When only two independent variables (our case) $x_{1}$ and $x_{2}$ are considered, the operator $\mathcal{L}$ applied to the function $u$ reads:
\begin{equation*}
    \mathcal{L}u = 
    A \cdot \dfrac{\partial^{2} u}{\partial x_{1}^{2}} +
    B \cdot \dfrac{\partial^{2} u}{\partial x_{1} \: \partial x_{2}} +
    C \cdot \dfrac{\partial^{2} u}{\partial x_{2}^{2}} +
    D \cdot \dfrac{\partial u}{\partial x_{1}} +
    E \cdot \dfrac{\partial u}{\partial x_{2}} +
    F \cdot u
\end{equation*}
For some constant coefficients $A, B, C, D, E, F, G \in \mathbb{R}$. If $d = 2$ (our case), the \textbf{independent variables} can represent the \emph{space coordinates}:
\begin{itemize}
    \item $x_{1} = x$
    \item $x_{2} = y$
\end{itemize}
After introducing the \definition{PDE discriminant} (a quantity that helps determine the type of PDE):
\begin{equation}
   \Delta := B^{2} - 4AC 
\end{equation}
\newpage
\noindent
The PDE can be classified as:
\begin{itemize}
    \item \definition{Elliptic PDE} if $\Delta < 0$
    \item \definition{Parabolic PDE} if $\Delta = 0$
    \item \definition{Hyperbolic PDE} if $\Delta > 0$
\end{itemize}

\begin{flushleft}
    \textcolor{Green3}{\faIcon{question-circle} \textbf{What are the implications of PDE classification?}}
\end{flushleft}
The different nature of the PDE impacts on:
\begin{itemize}
    \item \textbf{Type} and \textbf{amount of data to prescribe as boundary};
    \item \textbf{Initial conditions} to ensure the well-posedness of the problem (existence and uniqueness of the solution);
    \item The \textbf{phenomena that can be described} by the PDE;
    \item The \textbf{information that encapsulates}.
\end{itemize}
In general:
\begin{itemize}
    \item \textbf{Elliptic PDE} typically describes \textbf{stationary phenomena}, without time evolution of quantities.
    \item \textbf{Parabolic PDE} describes \textbf{wave propagation phenomena} with \underline{infinite} velocity of propagation.
    \item \textbf{Hyperbolic PDE} describes \textbf{wave propagation phenomena} but with \underline{finite} velocity of propagation.
\end{itemize}