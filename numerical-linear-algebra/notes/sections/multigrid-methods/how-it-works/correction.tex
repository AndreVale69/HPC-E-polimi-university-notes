\subsubsection{Correction}\label{subsubsection: Correction}

\textbf{\underline{Purpose}}. It is a critical part of the process that \textbf{ensures efficient error reduction across multiple grid levels}. The steps are as follows:
\begin{enumerate}
    \item \textbf{\underline{Pre-Smoothing}}. Relax $\nu_{1}$ times on $A_{h}\mathbf{x}_{h} = \mathbf{b}_{h}$ to obtain an approximation $\mathbf{x}_{h}^{\left(k+\nu_{1}\right)}$.
    
    This step aims to \textbf{reduce high frequency errors on the fine grid}. Smoothing (relaxation) techniques such as Gauss-Seidel or Jacobi iterations are typically used.
    
    \item \textbf{\underline{Compute Residual}}. Compute $\mathbf{r}_{h}^{\left(k+\nu_{1}\right)} = \mathbf{b}_{h} - A_{h}\mathbf{x}_{h}^{\left(k+\nu_{1}\right)}$
    
    The residual represents the error in the current solution and is used to determine the correction required.

    \item \textbf{\underline{Restriction to Coarse Grid}}. Move the residual $\mathbf{r}_{h}^{\left(k+\nu_{1}\right)}$ from (the fine grid) $\mathcal{F}_{h}$ to (the coarse grid) $\mathcal{F}_{2h}$ to obtain $\mathbf{r}_{2h}^{\left(k+\nu_{1}\right)}$.

    This step transfers the error information to a coarser grid where it is easier to handle low frequency errors.

    \item \textbf{\underline{Solve on Coarse Grid}}. Solve the residual equations $A_{2h}\mathbf{e}_{2h} = \mathbf{r}_{2h}^{\left(k+\nu_{1}\right)}$ on $\mathcal{F}_{2h}$. Where $\mathbf{e}_{2h}$ is the error estimate.

    Coarse grid solving addresses the lower frequency errors that are more difficult to smooth on the fine grid.

    \item \textbf{\underline{Prolongation to Fine Grid}}. Move the error calculated previously $\mathbf{e}_{2h}$ from (the coarse grid) $\mathcal{F}_{2h}$ to (the fine grid) $\mathcal{F}_{h}$ to obtain $\mathbf{e}_{h}$.

    This step transfers the correction back to the fine grid, where it can be applied to improve the solution.

    \item \textbf{\underline{Correction}}. Correct the approximation obtained on (the fine grid) $\mathcal{F}_{h}$ with the error estimate obtained on (the coarse grid) $\mathcal{F}_{2h}$, i.e., $\mathbf{x}_{h}^{\left(k+1\right)} = \mathbf{x}_{h}^{\left(k+\nu_{1}\right)} + \mathbf{e}_{h}$.

    Applying this correction refines the solution on the fine grid, reducing the overall error.
\end{enumerate}
We can summarize these steps as follows:
\begin{enumerate}
    \item \textbf{Pre-Smoothing}. Reduces high-frequency errors on the fine grid.

    \item \textbf{Compute Residual}. Identifies remaining errors.
    
    \item \textbf{Coarse Grid Correction}. Targets lower-frequency errors by solving on a coarser grid.
    
    \item \textbf{Prolongation and Correction}. Transfers and applies corrections to refine the fine grid solution.
    
    \item \textbf{Post-Smoothing}. Further smooths any remaining errors.
\end{enumerate}