\subsection{How it works}

The multigrid method is divided into \textbf{seven parts} that make the MG method work.
\begin{enumerate}
    \item \textbf{Coarse Grids} (page \pageref{subsubsection: Coarse Grids})
    \item \textbf{Correction} (page \pageref{subsubsection: Correction})
    \item \textbf{Interpolation Operator} (page \pageref{subsubsection: Interpolation Operator})
    \item \textbf{Restriction Operator} (page \pageref{subsubsection: Restriction Operator})
    \item \textbf{Two-Grid Scheme} (page \pageref{subsubsection: Two-Grid Scheme})
    \item \textbf{V-Cycle Scheme} (page \pageref{subsubsection: V-Cycle Scheme})
\end{enumerate}

\noindent
These elements \textbf{work together to handle errors at different scales}, making the method \textbf{highly effective for solving large and complex systems} of linear equations.

\highspace
Note that this \textbf{is \underline{not} an algorithm!} We can think of the MG method as a toolbox filled with powerful tools, each designed to address different aspects of solving complex problems efficiently.

\begin{flushleft}
    \textcolor{Green3}{\faIcon{square-root-alt} \textbf{Notation used in MG methods}}
\end{flushleft}
We will use the subscript $h$ to indicate the Grid Spacing. The variable $h$ represents the \textbf{distance between two successive grid points on the fine grid}. For example, if the domain is divided into $N$ intervals, the grid spacing $h$ is typically $\frac{1}{N}$.
\begin{itemize}
    \item \textbf{Residual} $\mathbf{r}_{h}$ represents the residual calculated on the fine grid with spacing $h$. It's the difference between the current solution and the exact solution on this grid.

    \item \textbf{Solution} $\mathbf{x}_{h}$ indicates the approximate solution on the fine grid. This solution is updated iteratively using the Multigrid method.
    
    \item \textbf{Operator} $A_{h}$ is the matrix or operator that represents the system of equations on the fine grid. This operator acts on the solution $\mathbf{x}_{h}$.
    
    \item \textbf{Right-Hand Side} $\mathbf{b}_{h}$ is the right-hand side vector of the system of equations on the fine grid. It's what the solution $\mathbf{x}_{h}$ should ideally satisfy when acted upon by $A_{h}$.
    
    \item \textbf{Error on the Fine Grid} $\mathbf{e}_{h}$ is the error estimate or correction term calculated on the fine grid. It represents the difference between the true solution and the current approximate solution on the fine grid with spacing $h$.
\end{itemize}
What's more, \textbf{if we move to a coarser grid}, the grid spacing will be greater, indicated by $2h$ or even $4h$ if we make a significant jump to a coarser level.