\subsection{Matrix Operations}

Some basic matrix operations:
\begin{itemize}
	\item \textbf{Inner products}. If $\mathbf{x}, \mathbf{y} \in \mathbb{R}^{n}$ then:
	\begin{equation*}
		\mathbf{x}^{T} \mathbf{y} = \displaystyle\sum_{i = 1, \dots, n} x_{i}y_{i}
	\end{equation*}
	For real vectors, the commutative property is true:
	\begin{equation*}
		\mathbf{x}^{T} \mathbf{y} = \mathbf{y}^{T} \mathbf{x}
	\end{equation*}
	Furthermore, the vectors $\mathbf{x}, \mathbf{y} \in \mathbb{R}^{n}$ are \textbf{orthogonal} if:\index{Orthogonal Vectors}
	\begin{equation*}
		\mathbf{x}^{T} \mathbf{y} = \mathbf{y}^{T} \mathbf{x} = \mathbf{0}
	\end{equation*}
	And finally, some useful properties of matrix multiplication:
	\begin{enumerate}
		\item Multiplication by the \emph{identity} changes nothing.
		\begin{equation*}\index{Matrices Multiplication}
			A \in \mathbb{R}^{n \times m} \: \Rightarrow \: \mathbf{I}_{n} A = A = A\mathbf{I}_{m}
		\end{equation*}
		
		\item Associativity:
		\begin{equation*}\index{Matrix Associativity Property}
			A\left(BC\right) = \left(AB\right)C
		\end{equation*}
		
		\item Distributive:
		\begin{equation*}\index{Matrix Distributive Property}
			A\left(B+D\right) = AB + AD
		\end{equation*}
		
		\item \underline{No} commutativity:
		\begin{equation*}
			AB \ne BA
		\end{equation*}
		
		\item Transpose of product:
		\begin{equation*}\index{Transpose product between matrices}
			\left(AB\right)^{T} = B^{T} A^{T}
		\end{equation*}
	\end{enumerate}
	
	\item \textbf{Matrix powers}. For $A \in \mathbb{R}^{n \times n}$ with $A \ne \mathbf{0}$:
	\begin{equation*}
		A^{0} = \mathbf{I}_{n} \hspace{2em} A^{k} = \underbrace{A \cdots A}_{k\text{ times}} = AA^{k-1} \hspace{2em} k \ge 1
	\end{equation*}
	Furthermore, $A \in \mathbb{R}^{n \times n}$ is:
	\begin{itemize}
		\item \textbf{Idempotent} (projector) $A^{2} = A$ \index{Idempotent Matrices}
		\item \textbf{Nilpotent} $A^{k} = \mathbf{0}$ for some integer $k \ge 1$ \index{Nilpotent Matrices}
	\end{itemize}
	
	\item \textbf{Inverse}. For $A \in \mathbb{R}^{n \times n}$ is \definitionWithSpecificIndex{non-singular}{Non-singular Matrices} (\definitionWithSpecificIndex{invertible}{Invertible Matrices}), if exists $A^{-1}$ with:
	\begin{equation}\label{eq: non-singular matrix}
		AA^{-1} = \mathbf{I}_{n} = A^{-1}A
	\end{equation}
	Inverse and transposition are interchangeable:
	\begin{equation*}
		A^{-T} \triangleq \left(A^{T}\right)^{-1} = \left(A^{-1}\right)^{T}
	\end{equation*}
	Furthermore, an inverse of a product for a matrix $A \in \mathbb{R}^{n \times n}$ can be expressed as:
	\begin{equation*}
		\left(AB\right)^{-1} = B^{-1}A^{-1}
	\end{equation*}
	Finally, remark that if $\mathbf{0} \ne \mathbf{x} \in \mathbb{R}^{n}$ and $A\mathbf{x} = 0$, then $A$ is \definitionWithSpecificIndex{singular}{Singular Matrices}.
	
	\item \textbf{Orthogonal matrices}. Given a matrix $A \in \mathbb{R}^{n \times n}$ that is \emph{invertible}, the matrix $A$ is said to be \definitionWithSpecificIndex{orthogonal}{Orthogonal Matrices} if:
	\begin{equation*}
		A^{-1} = A^{T} \: \Rightarrow \: A^{T}A = \mathbf{I}_{n} = AA^{T}
	\end{equation*}
	
	\item \textbf{Triangular matrices}. There are two types of triangular matrices:
	\begin{enumerate}
		\item \definition{Upper triangular matrix}:
		\begin{equation*}
			\mathbf{U} = \begin{bmatrix}
				u_{1,1} & u_{1,2} & \cdots & u_{1,n} \\
				0 & u_{2,2} & \cdots & u_{2,n} \\
				\vdots & \cdots & \ddots & \vdots \\
				0 & 0 & \cdots & u_{n,n} \\
			\end{bmatrix}
		\end{equation*}
		$\mathbf{U}$ is \textbf{non-singular} if and only if $u_{ii} \ne 0$ for $i = 1, \dots, n$.

		\item \definition{Lower triangular matrix}:
		\begin{equation*}
			\mathbf{L} = \begin{bmatrix}
				l_{1,1} & 0 & \cdots & 0 \\
				l_{2,1} & l_{2,2} & \cdots & 0 \\
				\vdots & \cdots & \ddots & \vdots \\
				l_{n,1} & l_{n,2} & \cdots & l_{n,n} \\
			\end{bmatrix}
		\end{equation*}
		$\mathbf{L}$ is \textbf{non-singular} if and only if $l_{ii} \ne 0$ for $i = 1, \dots, n$.
	\end{enumerate}
	
	\item \textbf{Unitary triangular matrices}. Are matrices similar to the lower and upper matrices, but they have the main diagonal composed of ones.
	\begin{enumerate}
		\item \definition{Unitary upper triangular matrix}:
		\begin{equation*}
			\mathbf{U} = \begin{bmatrix}
				1 & u_{1,2} & \cdots & u_{1,n} \\
				0 & 1 & \cdots & u_{2,n} \\
				\vdots & \cdots & \ddots & \vdots \\
				0 & 0 & \cdots & 1 \\
			\end{bmatrix}
		\end{equation*}

		\item \definition{Unitary lower triangular matrix}:
		\begin{equation*}
			\mathbf{L} = \begin{bmatrix}
				1 & 0 & \cdots & 0 \\
				l_{2,1} & 1 & \cdots & 0 \\
				\vdots & \cdots & \ddots & \vdots \\
				l_{n,1} & l_{n,2} & \cdots & 1 \\
			\end{bmatrix}
		\end{equation*}
	\end{enumerate}
\end{itemize}