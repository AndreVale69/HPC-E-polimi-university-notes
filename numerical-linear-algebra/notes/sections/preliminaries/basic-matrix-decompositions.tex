\subsection{Basic matrix decomposition}

In the Numerical Linear Algebra course, we will use three main decomposition:
\begin{itemize}
	\item \underline{\textbf{LU factorization with (partial) pivoting}}. If $A \in \mathbb{R}^{n \times n}$ is a nonsingular matrix, then:
	\begin{equation*}
		PA = LU
	\end{equation*}
	Where:
	\begin{itemize}
		\item $P$ is a permutation matrix
		\item $L$ is an unit lower triangular matrix
		\item $U$ is an upper triangular matrix
	\end{itemize}
	Note that the linear system solution:
	\begin{equation*}
		A\mathbf{x} = \mathbf{b}
	\end{equation*}
	Can be solved directly by calculation:
	\begin{equation*}
		PA = LU
	\end{equation*}
	This way the complexity is equal to $O\left(n^{3}\right)$. So a smarter way to reduce complexity is to use the \emph{divide et impera} (or \emph{divide and conquer}) technique. Then solve the system:
	\begin{equation*}
		\begin{cases}
			L\mathbf{y} = P\mathbf{b} & \rightarrow \text{ unit lower triangular system, complexity } O\left(n^{2}\right) \\
			U\mathbf{x} = \mathbf{y}  & \rightarrow \text{ upper triangular system, complexity } O\left(n^{2}\right)
		\end{cases}
	\end{equation*}

	\item \underline{\textbf{Cholesky decomposition}}. If $A \in \mathbb{R}^{n \times n}$ is a symmetric\footnote{$A^{T} = A$} and positive definite\footnote{$\mathbf{z}^{T} A \mathbf{z} > 0 \hspace{2em} \forall \mathbf{z} \ne 0$}, then:
	\begin{equation*}
		A = L^{T}L
	\end{equation*}
	Where $L$ is a lower triangular matrix (with positive entries on the diagonal). Also note that the linear system solution:
	\begin{equation*}
		A\mathbf{x} = \mathbf{b}
	\end{equation*}
	Can be solved directly by calculation:
	\begin{equation*}
		A = L^{T}L
	\end{equation*}
	This way the complexity is equal to $O\left(n^{3}\right)$. So a smarter way to reduce complexity is to use the \emph{divide et impera} (or \emph{divide and conquer}) technique. Then solve the system:
	\begin{equation*}
		\begin{cases}
			L^{T}\mathbf{y} = \mathbf{b} & \rightarrow \text{ lower triangular system, complexity } O\left(n^{2}\right) \\
			L\mathbf{x} = \mathbf{y}  & \rightarrow \text{ upper triangular system, complexity } O\left(n^{2}\right)
		\end{cases}
	\end{equation*}

	\item \underline{\textbf{QR decomposition}}. If $A \in \mathbb{R}^{n \times n}$ is a nonsingular matrix, then:
	\begin{equation*}
		A = QR
	\end{equation*}
	Where:
	\begin{itemize}
		\item $Q$ is an orthogonal matrix
		\item $R$ is an upper triangular
	\end{itemize}
	Note that the linear system solution:
	\begin{equation*}
		A\mathbf{x} = \mathbf{b}
	\end{equation*}
	Can be solved directly by calculation:
	\begin{equation*}
		A = QR
	\end{equation*}
	This way the complexity is equal to $O\left(n^{3}\right)$. So a smarter way to reduce complexity is to use the \emph{divide et impera} (or \emph{divide and conquer}) technique. Then:
	\begin{enumerate}
		\item Multiply $\mathbf{c} = Q^{T}\mathbf{b}$, complexity $O\left(n^{2}\right)$
		
		\item Solve the lower triangular system $R\mathbf{x} = \mathbf{c}$, complexity $O\left(n^{2}\right)$
	\end{enumerate}
\end{itemize}