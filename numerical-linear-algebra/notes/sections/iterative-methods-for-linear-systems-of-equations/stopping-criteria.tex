\subsection{Stopping Criteria}

A practical test is needed to determine when to stop the iteration. The \textbf{main idea} is that we stop iterations when:
\begin{equation*}
    \dfrac{\left|\left|\mathbf{x} - \mathbf{x^{\left(k\right)}}\right|\right|}{\left|\left|\mathbf{x}^{\left(k\right)}\right|\right|} \le \varepsilon
\end{equation*}
Where $\varepsilon$ is a \textbf{user defined tolerance}. Meanwhile, the error (left side of the equation) is unknown! There are two criteria we can use to replace it:
\begin{itemize}
    \item \definition{Residual-based stopping criteria}. It looks at the \emph{residual}, which is the difference between the current solution and the one obtained by reapplying the method's equation:
    \begin{equation*}
        r^{\left(k\right)} = b - Ax^{\left(k\right)}
    \end{equation*}
    This residual gets smaller as the solution gets closer to the exact answer. When it's small enough, the iteration stops. This approach works because the residual essentially tracks the behaviour of the error. When the residual is small, the error is usually small.
    
    From a mathematical point of view:
    \begin{equation*}
        \dfrac{\left|\left|\mathbf{x} - \mathbf{x^{\left(k\right)}}\right|\right|}{\left|\left|\mathbf{x}^{\left(k\right)}\right|\right|} \le K\left(A\right) \dfrac{\left|\left|\mathbf{r}^{\left(k\right)}\right|\right|}{\left|\left|\mathbf{b}\right|\right|} \Longrightarrow \dfrac{\left|\left|\mathbf{r}^{\left(k\right)}\right|\right|}{\left|\left|\mathbf{b}\right|\right|} \le \varepsilon
    \end{equation*}
    Where $K\left(A\right)$ is the \definitionWithSpecificIndex{condition number}{Condition Number} of $A$. It is a measure of \textbf{how sensitive the solution of a system of linear equations is to errors in the data or errors in the solution process}.
    \begin{itemize}
        \item A \textbf{low condition number} (close to 1) means that the matrix is well conditioned, and \textbf{small errors in the data will cause only small errors in the solution}.

        \item A \textbf{high condition number} indicates that the matrix is poorly conditioned, and even \textbf{small errors in the data can lead to large errors in the solution}.
    \end{itemize}
    To reduce the condition number and the error, we need to use a preconditioner on the main matrix $A$. So instead of solving the general problem $Ax = b$ directly, we choose a preconditioner $P$ and solve $P^{-1}Ax = P^{-1}b$:
    \begin{equation*}
        \dfrac{\left|\left|\mathbf{x} - \mathbf{x^{\left(k\right)}}\right|\right|}{\left|\left|\mathbf{x}^{\left(k\right)}\right|\right|} \le K\left(P^{-1} A\right) \dfrac{\left|\left|\mathbf{z}^{\left(k\right)}\right|\right|}{\left|\left|\mathbf{b}\right|\right|} \Longrightarrow
        \dfrac{\left|\left|\mathbf{z}^{\left(k\right)}\right|\right|}{\left|\left|\mathbf{b}\right|\right|} \le \varepsilon \hspace{2em} \mathbf{z}^{\left(k\right)} = P^{-1}\mathbf{r^{\left(k\right)}}
    \end{equation*}

    \item \definition{Distance between consecutive iterates criteria}. It looks at \textbf{how much the current iterate (solution) changes compared to the previous one. When this difference becomes small enough, it's a signal that the method is converging and can be stopped}.
    
    Mathematically, define:
    \begin{equation*}
        \mathbf{\delta}^{\left(k\right)} = \mathbf{x}^{\left(k+1\right)} - \mathbf{x}^{\left(k\right)} \Longrightarrow \left|\left|\mathbf{\delta}^{\left(k\right)}\right|\right| \le \varepsilon \Longrightarrow \left|\left|\mathbf{x}^{\left(k+1\right)} - \mathbf{x}^{\left(k\right)}\right|\right|
    \end{equation*}
    With some manipulation, we can also demonstrate the relation between the true error and $\delta^{\left(k\right)}$:
    \begin{equation*}
        \left|\left|\mathbf{e}^{\left(k\right)}\right|\right| \le \dfrac{1}{1 - \rho\left(B\right)} \cdot \left|\left|\mathbf{\delta}^{\left(k\right)}\right|\right|
    \end{equation*}
    Indeed:
    \begin{equation*}
        \begin{array}{rcl}
            \left|\left|\mathbf{e}^{\left(k\right)}\right|\right| &=& \left|\left|\mathbf{x}-\mathbf{x}^{\left(k\right)}\right|\right| \\ [.5em]
            %
            &=& \left|\left|\mathbf{x}-\mathbf{x}^{\left(k+1\right)} + \mathbf{x}^{\left(k+1\right)} - \mathbf{x}^{\left(k\right)}\right|\right| \\ [.5em]
            %
            &=& \left|\left|\mathbf{e}^{\left(k+1\right)} + \mathbf{\delta}^{\left(k\right)}\right|\right| \\ [.5em]
            %
            &\le& \rho\left(B\right) \cdot \left|\left|\mathbf{e}^{\left(k\right)}\right|\right| + \left|\left|\mathbf{\delta}^{\left(k\right)}\right|\right|
        \end{array}
    \end{equation*}
\end{itemize}