\subsection{Krylov-space}

Krylov space methods are a group of iterative techniques used to solve large linear systems or eigenvalue problems. These methods construct a sequence of subspaces, called Krylov subspaces, which are iteratively expanded to approximate the solution.

\begin{definitionbox}[: Krylov (sub)space]
    Given a nonsingular $A \in \mathbb{R}^{n \times n}$ and $\mathbf{y} \in \mathbb{R}^{n}$, $\mathbf{y} \ne \mathbf{0}$, the $k$th Krylov (sub)space $\mathcal{K}_{k}\left(A, \mathbf{y}\right)$ generated by $A$ from $\mathbf{y}$ is:
    \begin{equation}
        \mathcal{K}_{k}\left(A, \mathbf{y}\right) = \mathrm{span}\left(\mathbf{y}, A\mathbf{y}, \dots, A^{k-1}\mathbf{y}\right)
    \end{equation}
    Clearly, it holds:
    \begin{equation*}
        \mathcal{K}_{1}\left(A, \mathbf{y}\right)
        \subseteq
        \mathcal{K}_{2}\left(A, \mathbf{y}\right)
        \subseteq
        \cdots
    \end{equation*}
\end{definitionbox}

\noindent
It seems clever to choose the $k$th approximate solution $\mathbf{x}^{\left(k\right)}$:
\begin{equation*}
    \mathbf{x}^{\left(k\right)} \in \mathbf{x}^{\left(0\right)} + \mathcal{K}_{k}\left(A, \mathbf{r}^{\left(0\right)}\right)
\end{equation*}
But can we expect to find the exact solution $\mathbf{x}$ of $A\mathbf{x} = \mathbf{b}$ in one of those affine space?
\begin{lemma}
    Let $\mathbf{x}$ be the solution of $A\mathbf{x} = \mathbf{b}$ and let $\mathbf{x}^{\left(0\right)}$ be any initial approximation of it and $\mathbf{r}^{\left(0\right)} = \mathbf{b} - A\mathbf{x}^{\left(0\right)}$ the corresponding residual. Moreover, let $v = v\left(\mathbf{r}^{\left(0\right)}, A\right)$ be the so called \textbf{grade of $\mathbf{r}^{\left(0\right)}$ with respect to $A$}. Then:
    \begin{equation*}
        \mathbf{x} \in \mathbf{x}^{\left(0\right)} + \mathcal{K}_{v}\left(A, \mathbf{r}^{\left(0\right)}\right)
    \end{equation*}
\end{lemma}

\begin{lemma}
    There is a positive integer $\nu = \nu\left(\mathbf{r}^{\left(0\right)}, A\right)$ called \textbf{grade of $\mathbf{y}$ with respect to $A$}, such that:
    \begin{equation*}
        \begin{array}{rcl}
            \dim\left(\mathcal{K}_{s}\left(A, y\right)\right) &=& s \text{ if } s \le \nu\\ [.5em]
            \dim\left(\mathcal{K}_{s}\left(A, y\right)\right) &=& \nu \text{ if } s \ge \nu
        \end{array}
    \end{equation*}
    $\mathcal{K}_{\nu}\left(A, y\right)$ is the smallest $A$-invariant subspace that contains $\mathbf{y}$.
\end{lemma}

\begin{lemma}
    The nonnegative integer $\nu = \nu\left(\mathbf{y}, A\right)$ of $\mathbf{y}$ with respect to $A$ satisfies:
    \begin{equation*}
        \nu\left(\mathbf{y}, A\right) = \min\left\{
            s \: \left| \: A^{-1}\mathbf{y} \in \mathcal{K}_{s}\left(A, y\right) \right.
        \right\}
    \end{equation*}
\end{lemma}

\noindent
The idea behind Krylov space solvers is to \textbf{generate a sequence of approximate solutions} $\mathbf{x}^{\left(k\right)} \in \mathbf{x}^{\left(0\right)} + \mathcal{K}_{k}\left(A, \mathbf{r}^{\left(0\right)}\right)$ of $A\mathbf{x} = \mathbf{b}$ so that the corresponding \textbf{residuals} $\mathbf{r}^{\left(k\right)} \in \mathcal{K}_{k+1}\left(A, \mathbf{r}^{\left(0\right)}\right)$ \textbf{\emph{converge} to the zero vector} $\mathbf{0}$.

\highspace
The \emph{converge} may also \textbf{mean that after a finite number of steps}, $\mathbf{r}^{\left(k\right)} = \mathbf{0}$, so that $\mathbf{x}^{\left(k\right)} = \mathbf{x}$ and the process stops. This is especially true (in exact arithmetic) if a \textbf{method ensures that the residuals are linearly independent}: then $\mathbf{r}^{\left(\nu\right)} = \mathbf{0}$. In this case, we say that the \textbf{method has the property of finite termination}.

\begin{definitionbox}[: (standard) Krylov space]
    A (standard) Krylov space method for solving a linear system $A\mathbf{x} = \mathbf{b}$ or, briefly, a Krylov space solver is an iterative method starting from some initial approximation $\mathbf{x}^{\left(0\right)}$ and the corresponding residual $\mathbf{r}^{\left(0\right)}$ and generating for all, or at least most $k$, until it possibly finds the exact solution, iterates $\mathbf{x}^{\left(k\right)}$ such that:
    \begin{equation}
        \mathbf{x}^{\left(k\right)} = \mathbf{x}^{\left(0\right)} + p_{k-1} \left(A\right)\mathbf{r}^{\left(0\right)}
    \end{equation}
    With a polynomial $p_{k-1}\left(A\right)$ of exact degree $k-1$. For some $k$, $\mathbf{x}^{\left(k\right)}$ may not exist or $p_{k-1}\left(A\right)$ may have lower degree.
\end{definitionbox}

\noindent
The conjugate gradient method is a Krylov space solver.

\highspace
\textbf{Solving nonsymmetric linear systems iteratively with Krylov space solvers is considerably more difficult and costly than symmetric systems}. There are two different ways to generalize the Conjugate Gradient:
\begin{itemize}
    \item Maintain the orthogonality of the projection and the related minimality of the error by constructing either orthogonal residuals $\mathbf{x}^{\left(k\right)}$. Then, the recursions involve all previously constructed residuals or search directions and all previously constructed iterates.
    \item (Preferred) Maintain short recurrence formulas for residuals, direction vectors and iterates (BiConjugate Gradient (BiCG) method, Lanczos-type product methods (LTPM)). The resulting methods are at best oblique projection methods. There is no minimality property of error or residuals vectors.
\end{itemize}