\subsubsection{BiConjugate Gradient (BiCG) and BiCGSTAB method}

The \definition{BiConjugate Gradient (BiCG) method} is an iterative algorithm used to \textbf{solve non-symmetric linear systems of equations}, $A\mathbf{x} = \mathbf{b}$. It \textbf{extends the Conjugate Gradient (CG) method to handle matrices that are not symmetric or positive definite}.

\highspace
BiCG has the peculiarity of \textbf{simultaneously solving} the original system $A\mathbf{x} = \mathbf{b}$ (where $A$ is a square matrix and $\mathbf{x}, \mathbf{b}$ are column vectors) and a dual system $\widehat{\mathbf{x}} A^{T} = \widehat{\mathbf{b}}$ (where the $A^{T} \ne A$ and $\widehat{\mathbf{x}}, \widehat{\mathbf{b}}$ are row vectors).

\highspace
While CG has mutually orthogonal residual $\mathbf{r}^{\left(k\right)}$, BiCG constructs in the same spaces residuals that are orthogonal to a dual Krylov space spanned by \dquotes{shadow residuals}:
\begin{equation}
    \begin{array}{rcl}
        \tilde{\mathbf{r}}^{\left(k\right)} &=& p_{k}\left(A^{T}\right) \tilde{\mathbf{r}}^{\left(0\right)} \in \mathrm{span} \left\{ \tilde{\mathbf{r}}^{\left(0\right)}, A^{T}\tilde{\mathbf{r}}^{\left(0\right)}, \dots, \left(A^{T}\right)^{k} \tilde{\mathbf{r}}^{\left(0\right)} \right\} \\ [1em]
        &=& \mathcal{K}_{k+1} \left(A^{T}, \tilde{\mathbf{r}}^{\left(0\right)}\right) \equiv \tilde{\mathcal{K}}_{k+1}
    \end{array}
\end{equation}
The initial shadow residual $\tilde{\mathbf{r}}^{\left(0\right)}$ can be chosen freely. So, BiCG requires two matrix-vector multiplications to extend $\mathcal{K}_{k}$ and $\tilde{\mathcal{K}}_{k}$: one multiplication by $A$ (the original system) and one by $A^{T}$ (the dual system).

\highspace
\begin{flushleft}
    \textcolor{Green3}{\faIcon{tools} \textbf{BiCG Algorithm}}
\end{flushleft}
\begin{enumerate}
    \item \textbf{Initial guess}. Start with an initial guess $\mathbf{x}^{\left(0\right)}$ (column vector), $\widehat{\mathbf{x}}^{\left(0\right)}, \widehat{\mathbf{b}}$ (row vectors).
    
    \item \textbf{Compute initial residual}. Define the residual $\mathbf{r}^{\left(0\right)} = \mathbf{b} - A\mathbf{x}^{\left(0\right)}$ (column vector) and the shadow residual $\widehat{\mathbf{r}}^{\left(0\right)} = \widehat{\mathbf{b}} - \widehat{\mathbf{x}}^{\left(0\right)}A^{T}$ (row vector).
    
    \item \textbf{Initial direction}. The direction is equal to the residual $\mathbf{d}_{0} = \mathbf{r}^{\left(0\right)}$ (column vector), and the shadow direction is equal to the shadow residual $\widehat{\mathbf{d}}_{0} = \widehat{\mathbf{r}}^{\left(0\right)}$ (row vector).

    \item \textbf{Iteration}. Continue to iterate until the stopping criteria is met:
    \begin{enumerate}
        \item Parameter $\alpha_{k}$:
        \begin{equation*}
            \alpha_{k} = \dfrac{
                \widehat{\mathbf{r}}^{\left(k\right)} \mathbf{r}^{\left(k\right)}
            }{
                \widehat{\mathbf{d}}_{k} A \mathbf{d}_{k}
            }
        \end{equation*}

        \item Update the solution for both systems:
        \begin{equation*}
            \begin{array}{rcl}
                \mathbf{x}^{\left(k+1\right)} &=& \mathbf{x}^{\left(k\right)} + \alpha_{k}\mathbf{d}_{k} \\ [.5em]
                \widehat{\mathbf{x}}^{\left(k+1\right)} &=& \widehat{\mathbf{x}}^{\left(k\right)} + \alpha_{k}\widehat{\mathbf{d}}_{k}
            \end{array}
        \end{equation*}

        \item Update the residual for both systems:
        \begin{equation*}
            \begin{array}{rcl}
                \mathbf{r}^{\left(k+1\right)} \left(\equiv \mathbf{b} - A \mathbf{x}^{\left(k+1\right)}\right) &=& \mathbf{r}^{\left(k\right)} + \alpha_{k}A\mathbf{d}_{k} \\ [.5em]
                \widehat{\mathbf{r}}^{\left(k+1\right)} \left(\equiv \widehat{\mathbf{b}} - \widehat{\mathbf{x}}^{\left(k+1\right)} A^{T} \right) &=& \widehat{\mathbf{r}}^{\left(k\right)} - \alpha_{k}\widehat{\mathbf{d}}_{k} A^{T}
            \end{array}
        \end{equation*}

        \item Parameter $\beta_{k}$:
        \begin{equation*}
            \alpha_{k} = \dfrac{
                \widehat{\mathbf{r}}^{\left(k+1\right)} \mathbf{r}^{\left(k+1\right)}
            }{
                \widehat{\mathbf{r}}^{\left(k\right)} \mathbf{r}^{\left(k\right)}
            }
        \end{equation*}

        \item Update the direction:
        \begin{equation*}
            \begin{array}{rcl}
                \mathbf{d}_{k+1}            &=& \mathbf{r}^{\left(k+1\right)} + \beta_{k} + \mathbf{d}_{k} \\ [.5em]
                \widehat{\mathbf{d}}_{k+1}  &=& \widehat{\mathbf{r}}^{\left(k+1\right)} + \beta_{k} + \widehat{\mathbf{d}}_{k}
            \end{array}
        \end{equation*}
    \end{enumerate}
\end{enumerate}
In practice the $\widehat{\mathbf{x}}^{\left(0\right)} = \left[\mathbf{x}^{\left(0\right)}\right]^{T}$ and $\widehat{\mathbf{b}} = \mathbf{b}^{T}$. We also need to make sure that $\widehat{\mathbf{r}}^{\left(0\right)}\mathbf{r}^{\left(0\right)} \ne 0$.

\begin{flushleft}
    \textcolor{Red2}{\faIcon{dollar-sign} \textbf{How much does it cost and why do we need to use BiCGSTAB?}}
\end{flushleft}
Each iteration costs twice as much as a CG iteration:
\begin{itemize}
    \item \textbf{Dense matrix}: the cost of each iteration is about $2n^{2}$ \textbf{operations}.
    \item \textbf{Sparse matrix}: the cost of each iteration is only about $2n$ \textbf{operations}.
\end{itemize}
It also has a \textbf{big problem}: \textbf{numerical stability}. BiCG uses duality, which introduces a level of complexity that can lead to numerical instability, especially because of the \textbf{multiplication of} $A$ and $A^{T}$. Fortunately, the \definition{BiConjugate Gradient Stabilized (BiCGSTAB) method} is a variant of BiCG and has \textbf{faster and smoother convergence than the original BiCG}. The main idea in BiCGSTAB is not to keep track of residuals and search directions, but to use techniques to stabilize the convergence and improve the robustness of the method.

\highspace
\begin{flushleft}
    \textcolor{Green3}{\faIcon{network-wired} \textbf{Can it be parallelized?}}
\end{flushleft}
BiCGSTAB can be \textbf{implemented on GPUs using frameworks like CUDA}. This allows for massive parallelism, as GPUs have thousands of cores that can perform computations simultaneously. BiCGSTAB can also be \textbf{parallelized on distributed memory systems using MPI} (Message Passing Interface). This involves partitioning the matrix and distributing the computations across multiple processors. The communication between processors is managed efficiently to minimize overhead and maximize performance.