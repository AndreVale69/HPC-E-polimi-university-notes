\subsection{Pivoting}

\begin{flushleft}
    \textcolor{Red2}{\faIcon{exclamation-triangle} \textbf{Problem}}
\end{flushleft}
If, during the $k$-th step of Gaussian elimination, the \textbf{pivot element} $a_{kk}^{(k)}$ \textbf{is zero}, we cannot proceed with the standard elimination process because division by zero is undefined.

\highspace
\begin{flushleft}
    \textcolor{Green3}{\faIcon{check-circle} \textbf{Solution}}
\end{flushleft}
\definition{Pivoting} is a \textbf{set of techniques used to prevent division by zero in Gaussian elimination}, but also to \textbf{provide numerical stability} and \textbf{minimize rounding errors}. There are several types of pivoting: \emph{partial pivoting} (pivoting by rows), \emph{complete pivoting}.

\highspace
The pivoting strategy can be interpreted as pre-multiplying the original matrix $A$ (and the constants vector $b$) by a permutation matrix $P$. The permutation matrix $P$ reorders the rows of $A$ and $b$ to place a suitable pivot element in the $k$-th row.
\begin{equation*}
    A\mathbf{x} = \mathbf{b} \: \Rightarrow \: PA\mathbf{x} = P\mathbf{b} \: \Rightarrow \: LU\mathbf{x} = P\mathbf{b} \: \Rightarrow \: \begin{cases}
        L\mathbf{y} = P\mathbf{b} \\
        U\mathbf{x} = \mathbf{y}
    \end{cases}
\end{equation*}