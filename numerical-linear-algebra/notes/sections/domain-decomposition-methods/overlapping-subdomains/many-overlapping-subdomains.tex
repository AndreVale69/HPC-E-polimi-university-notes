\subsubsection{Many Overlapping Subdomains}

When applying the Schwarz method, using many overlapping subdomains can increase parallelism. This approach can be more efficient than just using a two-domain algorithm, especially for large-scale problems.

\highspace
To achieve a \textbf{higher degree of parallelism} with the Schwarz method, we can \textbf{apply the two-domain algorithm recursively or use many subdomains}. If there are $N$ overlapping subdomains, we define matrices $R_{i}$ and $A_{i}$ for each subdomain $i$ (where $i = 1, \ldots, N$). The \textbf{Additive Schwarz preconditioner} then takes the form:
\begin{equation}
    P_{ad}^{-1} = \displaystyle\sum_{i=1, \ldots, N} R_{i}^{T} A_{i}^{-1} R_{i}
\end{equation}

\highspace
\begin{flushleft}
    \textcolor{Green3}{\faIcon{book} \textbf{Generalization and Convergence Issues}}
\end{flushleft}
The \textbf{generalization of the block-Jacobi iteration} to many subdomains is highly parallel but \textbf{not algorithmically scalable because the convergence rate degrades as $N$ increases}. To restore the convergence rate, a \textbf{coarse grid correction is used to provide global coupling}. The updated \textbf{Additive Schwarz preconditioner}, including the coarse grid correction, takes the form:
\begin{equation}
    P_{ad} = \displaystyle\sum_{i=0, \ldots, N} R_{i}^{T} A_{i}^{-1} R_{i}
\end{equation}

\highspace
\begin{flushleft}
    \textcolor{Green3}{\faIcon{book} \textbf{Multiplicative Schwarz Iteration}}
\end{flushleft}
The Multiplicative Schwarz iteration for $p$ domains is defined similarly to the two-domain case but extended to $p$ subdomains. Like the classical Gauss-Seidel method (compared to Jacobi), the \textbf{Multiplicative Schwarz method has a faster convergence rate than the corresponding Additive Schwarz method}. However, it \textbf{still requires a coarse grid correction to remain scalable}. A \textbf{limitation} of the Multiplicative Schwarz method \textbf{is the lack of parallelism since $p$ subproblems must be solved sequentially per iteration}. Parallelism can be introduced by \emph{coloring} subdomains to identify independent subproblems that can be solved simultaneously, similar to how parallelism is introduced in Gauss-Seidel.
