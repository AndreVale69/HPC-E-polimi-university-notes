\section{Domain Decomposition Methods}

\subsection{Introduction}

\definition{Domain Decomposition Methods (DDM)} are \textbf{numerical techniques used to solve large-scale computational problems by breaking them into smaller, more manageable subproblems}. These methods are essential in scientific computing, engineering simulations, and various other fields that require solving extensive linear systems or partial differential equations (PDEs).

\highspace
\begin{flushleft}
    \textcolor{Green3}{\faIcon{question-circle} \textbf{What Are Domain Decomposition Methods?}}
\end{flushleft}
DDM involves dividing a large \textbf{computational domain into smaller subdomains}. These subdomains are then \textbf{solved independently}, often \textbf{in parallel}, and their \textbf{solutions are combined to form the overall solution to the original problem}. This approach is particularly useful for problems that are too large to be solved as a single system due to computational limitations.

\highspace
\begin{flushleft}
    \textcolor{Red2}{\faIcon{exclamation-triangle} \textbf{Importance of Domain Decomposition Methods}}
\end{flushleft}
\begin{enumerate}
    \item \important{Parallelism}: By solving subdomains in parallel, DDM significantly reduces the computation time, making it \textbf{feasible to tackle massive problems} that would otherwise be intractable.
    \item \important{Scalability}: These methods can handle extremely large problems, ensuring that \textbf{computational resources are used efficiently}, and allowing for the \textbf{solution of problems on supercomputers or distributed computing systems}.
    \item \important{Modularity}: Breaking down a complex problem into smaller subproblems makes it \textbf{easier to manage}, \textbf{understand}, and \textbf{solve}. This modularity also facilitates debugging and improving algorithms.
    \item \important{Flexibility}: DDM can be \textbf{applied to various types of problems} across different disciplines, including fluid dynamics, structural mechanics, and electromagnetic simulations. This versatility makes them a powerful tool in the computational scientist's toolkit.
    \item \important{Improved Convergence}: With appropriate preconditioners and iterative methods, DDM can enhance the convergence rates of solving systems, leading to \textbf{faster and more accurate solutions}.
\end{enumerate}
