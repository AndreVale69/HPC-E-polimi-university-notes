\subsubsection{Deflation method}

\definitionWithSpecificIndex{Deflation}{Deflation method} is a technique used in conjunction with the Power Method to \textbf{find multiple eigenvalues and eigenvectors of a matrix}. This approach helps isolate and find successive eigenvalues by progressively "deflating" the influence of previously found eigenpairs.

\highspace
\begin{flushleft}
    \textcolor{Green3}{\faIcon{square-root-alt} \textbf{Mathematical point of view}}
\end{flushleft}
Suppose we have computed an eigenvalue $\lambda_{1}$ and corresponding eigenvector $\mathbf{v}_{1}$ (eigenpair) for a matrix $A$. We can compute additional eigenvalues $\lambda_{2}, \dots, \lambda_{n}$ of $A$ using deflation, which removes the known eigenvalue. The \emph{main idea} is: construct a new matrix $B$ with eigenvalues $\lambda_{2}, \dots, \lambda_{n}$, i.e. deflate the matrix $A$ by removing $\lambda_{1}$. Then $\lambda_{2}$ can be obtained by the power method.

\highspace
Now the interesting question is, how can we compute the new matrix $B$? We help us the similarity transformation. Let $S$ be any nonsingular matrix such that $S\mathbf{v}_{1} = \alpha\mathbf{e}_{1}$, that is $S$ is a scalar multiple of the first column $\mathbf{e}_{1}$ of the identity matrix $I$. Then, the similarity transformation determined by $S$ transforms $A$ into the form:
\begin{equation}
    SAS^{-1} = \begin{bmatrix}
        \lambda_{1} & b^{T} \\
        0 & B
    \end{bmatrix}
\end{equation}
We use $B$ to compute next eigenvalue $\lambda_{2}$ and eigenvector $\mathbf{z}_{2}$. Given $\mathbf{z}_{2}$ eigenvector of $B$, we want to compute the second eigenvector $\mathbf{v}_{2}$ of the matrix $A$. We need to add an element to vector $\mathbf{z}_{2}$ (that consist of $n - 1$ elements), that is
\begin{equation*}
    \mathbf{v}_{2} = S^{-1}\begin{pmatrix}
        \alpha \\ \mathbf{z}_{2}
    \end{pmatrix}
    \hspace{2em}
    \alpha = \dfrac{\mathbf{b}^{H} \mathbf{z}_{2}}{\lambda_{1} - \lambda_{2}}
\end{equation*}
Hence, $\mathbf{v}_{2}$ is an eigenvector corresponding to $\lambda_{2}$ for the original matrix $A$. The process can be repeated to find additional eigenvalues and eigenvectors.

\begin{flushleft}
    \textcolor{Green3}{\faIcon{tools} \textbf{Algorithm}}
\end{flushleft}
\begin{enumerate}
    \item \textbf{Find the Dominant Eigenvalue}. We use the Power Method to find the largest eigenvalue $\lambda_{1}$ and its corresponding eigenvector $\mathbf{v}_{1}$.
    
    \item \textbf{\emph{Deflate} the Matrix}. We modify the matrix to \emph{remove} the influence of the found eigenvalue and eigenvector.

    \item \textbf{Repeat}. Apply the Power Method to the deflated matrix to find the next largest eigenvalue.
\end{enumerate}
