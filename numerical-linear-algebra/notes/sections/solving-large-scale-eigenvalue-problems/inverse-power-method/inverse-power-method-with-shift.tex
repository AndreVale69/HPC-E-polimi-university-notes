\subsubsection{Inverse power method with shift}

The \definition{Inverse Power method with shift} \emph{extends} the standard inverse power method by improving convergence to certain eigenvalues near a chosen shift value $\mu$. This is particularly useful for \textbf{finding the eigenvalues closest to a given value}.

\highspace
\begin{flushleft}
    \textcolor{Green3}{\faIcon{tools} \textbf{Algorithm}}
\end{flushleft}
\begin{enumerate}
    \item \textbf{Start with an initial guess}, nonzero vector $\mathbf{q}^{\left(0\right)}$ such that its norm is one $\left|\left|\mathbf{q}^{\left(0\right)}\right|\right| = 1$.

    \textbf{Choose a shift} $\mu$ close to the desired eigenvalue.
    
    \textbf{Compose shifted matrix}:
    \begin{equation}
        M_{\mu} = A - \mu I
    \end{equation}

    \item \textbf{Iteration}. For each $k \ge 0$:
    \begin{enumerate}
        \item Solve the system:
        \begin{equation*}
            M_{\mu}\mathbf{z}^{\left(k+1\right)} = \mathbf{q}^{\left(k\right)}
        \end{equation*}

        \item After each system solution, normalize the vector to prevent it from growing too large:
        \begin{equation*}
            \mathbf{q}^{\left(k+1\right)} = \dfrac{
                \mathbf{z}^{\left(k+1\right)}
            }{
                \left|\left|\mathbf{z}^{\left(k+1\right)}\right|\right|
            }
        \end{equation*}

        \item Computes the Rayleigh quotient (see page \pageref{eq: Rayleigh quotient} for more details).
        \begin{equation*}
            \nu^{\left(k+1\right)} = \left[\mathbf{q}^{\left(k+1\right)}\right]^{H} A\mathbf{q}^{\left(k+1\right)}
        \end{equation*}
    \end{enumerate}
    \item \textbf{Repeat until we meet a specific stopping criteria}.
\end{enumerate}
We observe that the eigenvalue $\lambda$ of $A$ which is the closes to $\mu$ is the \textbf{minimum eigenvalue} of $M_{\mu}$.

\highspace
\begin{flushleft}
    \textcolor{Red2}{\faIcon{dollar-sign} \textbf{How much does it cost?}}
\end{flushleft}
It depends on the matrix used, the system to solve ($M_{\mu}\mathbf{z}^{\left(k+1\right)} = \mathbf{q}^{\left(k\right)}$) is the main cost:
\begin{itemize}
    \item \textbf{Dense matrix}. Each iteration costs $\approx n^{3}$ operations.
    \item \textbf{Sparse matrix}. Each iteration costs only $\approx n \cdot m$, where $n$ is the number of rows or columns of the square matrix and $m$ the number of non-zero elements.
\end{itemize}

\highspace
\begin{flushleft}
    \textcolor{Green3}{\faIcon{network-wired} \textbf{Can it be parallelized?}}
\end{flushleft}
The inverse power method with shift can be difficult to parallelize efficiently due to the nature of its iterative steps, but there are parts of the algorithm that can benefit from parallel processing. These include solving the linear system, normalization, and the Rayleigh quotient.